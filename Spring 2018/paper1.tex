Imagine that I promised John to deliver widgets to his factory but later
decided not to do so simply because I felt lazy. As a consequence of my
behavior, John's business suffered a tremendous loss. Surely morality
would call on me to do certain things to make up for the broken promise.
In Chapters 3 and 12 of \emph{The} \emph{Realm of Rights}\footnote{Judith
  Jarvis Thomson, \emph{The Realm of Rights}. (Cambridge, Massachusetts:
  Harvard Univ. Press, 1990).} Thomson notices the existence of such
moral norms in certain\footnote{I used ``certain''' here because in some
  cases, even though an agent fails to accord with her word, there is no
  moral residue generated. Thomson's account of moral residues is meant
  to cover cases where (1) an agent fails to accord with her word, and
  (2) there exist moral residues. (\emph{Thomson}, 85-86, 307-310) Thus,
  cases where there is no moral residue generated from failing to accord
  with one's word will not be discussed in this paper.} cases where an
agent fails to accord with her words. She calls such moral norms ``moral
residues''. Thomson contends that it is necessary to provide an account
for the existence of moral residues in order to fill in the gap between
the fact of breaking a promise and the fact of the existence of moral
residues. By examining intuitive cases of moral residues, Thomson
presents her account for the existence of moral residues in those cases,
which states that moral residues result from failing to fulfill a
certain type of claims.

Depending on the specific word-giving cases being examined, there might
be different accounts of moral residues. In this paper, I will limit my
discussion of moral residues to certain\footnote{Ibid.} cases of broken
promises. I will first explain Thomson's account for the existence of
moral residues by applying it to certain\footnote{Similarly, ``certain''
  here is meant to cover cases where there \emph{is} moral residue
  generated from failing to fulfill one's promise.} cases of an agent's
failing to fulfill a promise. I will then object to this explanation 1)
by pointing out that in some of such cases moral residues are in place
even when there is no Thomsonian claim, and 2) by arguing that the
Thomsonian account does not bridge the gap between the non-normative and
the normative. I will instead provide what I regard as the best account
of moral residues in certain cases of an agent failing to fulfill a
promise by appealing to what it means to make and break a promise.
Finally, I will contrast my account of moral residues with Thomson's to
show that my account avoids the problems that Thomson's account faces.

\section{What are Moral Residues?}

The notions of duty and moral requirement are essential for
understanding moral residues. Moral requirements are
all-things-considered moral oughts whereas duties are moral reasons to
do or not to do something. A duty is the correlative of a
claim.\footnote{\emph{Thomson}, 40} My duty towards you that I not eat
your lunch is correlative with your claim against me that I not eat your
lunch, and is a moral reason for me not to eat your lunch. However, I
might be morally required to eat your lunch considering other moral,
pragmatic, or epistemic factors such as preventing myself from starving
to death.

If I eat your lunch, then I infringe a claim of yours. I surely cannot
simply walk away and act as if nothing happened. It seems that I need to
carry out certain actions such as repaying you for your lunch to
compensate for any loss or harm associated with my infringing your
claim. This need for compensation is a moral residue resulting from me
doing something that I have a moral reason not to do, i.e. infringing
your claim against me. It seems that sometimes, when an agent does
something that she has a moral reason not to do, she is subjected to
moral residues. On the other hand, if I failed to deliver John widgets
even though I promised so, and if John suffered a tremendous loss from
my behavior, I am also subjected to moral residues for \emph{not} doing
something that I have a moral reason to do. Hence, when an agent does
\emph{not} do something that she has a moral reason to do, she is also
subjected to moral residues.

Having a general idea of what count as moral residues, Thomson roughly
defines moral residues as moral requirements that an agent is subjected
to due to her doing something that she has a moral reason not to do, or
vice versa.\footnote{\emph{Thomson}, 84} To say moral residues are moral
requirements is to say that moral residues are all-things-considered
oughts and that one must fulfill the content of moral residues \emph{no
matter what}. For instance, having the moral residue of repaying for
your lunch means that I am morally required to repay you for your lunch
\emph{no matter what}.

However, to say that moral residues are moral requirements is to make a
category mistake. Moral residues do not by themselves determine whether
an agent ought to do something, since it surely is a possibility that
there are other considerations (e.g. moral, epistemic, or pragmatic)
that are relevant to whether an agent ought to do something. Perhaps
some moral residues are trumped by other considerations, and an agent
ought \emph{not} to fulfill the content of the moral residue. It might
be the case that my moral residue of repaying for your lunch is trumped
by the consideration that I need to use all of my money to save a human
life. It would turn out that I ought \emph{not} to repay you for
something as trivial as lunch. Since it is fully possible for moral
residues to be trumped by other considerations, they cannot be moral
requirements.

However, moral residues do have constraining force on an agent insofar
as they are moral reasons relevant to whether an agent ought to do
something. This means that when one considers a course of action, one
needs to take into consideration the existence of moral residues. In the
above example, I have a moral residue to repay you for your lunch. When
I consider what I ought to do, I need to take into consideration the
existence of my moral residue and count it as a moral reason in favor of
the action of repaying you for your lunch.

Since moral residues are moral reasons to do or refrain from doing
something rather than moral requirements, I shall define moral residues
as duties that an agent is subjected to due to her doing something that
she has a moral reason not to do, or vice versa.

\section{Thomson's Account of the Origin of Moral Residues}

Moral residues are often in place when an agent fails to fulfill a
promise. Thomson defines a promise as such\footnote{\emph{Thomson},
  289-300}:

\begin{displayquote} \def\labelenumi{(\arabic{enumi})}
\itshape Y promises X that P, where P is an action or refraining from
action in the future or a limited range of states in the future, iff
	\begin{enumerate}
		\item acquires a duty of making it the case that P if it is possible
for her to make it the case that P, and
		\item if Y fails to make it the case that P through no fault of X's,
then Y obtains other duties, if it is possible for her to fulfill those
duties.
	\end{enumerate}
\end{displayquote}	
\noindent It is intuitive to think that Y's failing to fulfill a promise is doing
something that she has a moral reason not to do, and thus moral residues
would be generated by her failing to fulfill a promise.

\begin{displayquote} \def\labelenumi{(\arabic{enumi})}
\itshape For instance, I promise Danny that I will deliver coal to his
factory by Thursday, iff
	\begin{enumerate}
		\item I invite Danny to rely on the truth of the proposition ``I
will deliver coal to his factory by Thursday''; and
		\item Danny desires and accepts the invitation, and indeed does so,
and
		\item  Danny is the subject of my promise.
	\end{enumerate}
\end{displayquote}
\noindent Suppose all the conditions of promise-making obtained and I indeed made
a promise to Danny that I would deliver coal to his factory by Thursday.
However, unbeknownst to me, the only factory that supplies coal in the
city was bombed 10 minutes before I made the promise and I could not get
coal from anywhere else by Thursday. Due to such external circumstance,
I failed to fulfill the promise. My failing to fulfill this promise was
not doing something that I had a moral reason to do. It seems reasonable
that under such circumstance I am subjected to moral residues, the
contents of which might be

\begin{displayquote} \def\labelenumi{(\arabic{enumi})}
	``{\scshape MR1} - \textit{I deliver coal to his factory as soon as coal is
available'', and} \\
	
	``{\scshape MR2} - \textit{I compensate Danny for his financial loss}''.
\end{displayquote}

\noindent The above duties that I am subjected to due to my breaking the promise
are moral residues. Their existence is related to the fact that I broke
my promise to Danny. But exactly how are they related? It is a
non-normative fact that I broke a promise, but it is a normative fact
that I have those duties if it is indeed true that I have those duties.
How do we pass from the non-normative fact of the broken promise to the
normative fact of moral residues?

Thomson contends that there needs be some intermediary to bridge the
non-normative fact of the broken promise and the normative fact of the
moral residues.\footnote{\emph{Id}., 85} She proposes a possible
intermediary bridge between the non-normative fact and the normative
moral residues: if Y promises X that P, then X has a claim against Y
that P; if the infringement of X's claim leads to X's harm or loss, then
there exist moral residues.\footnote{\emph{Id}., 96} The existence of
moral residues come from the fact that a claim is infringed when one
breaks a promise.\footnote{\emph{Id}., 93-94} The reason to think that
such is the case is that a claim is equivalent to a constraint on the
claim-giver's behavior that includes such things as that the claim-giver
may have to make amends later if he or she does not accord the
claim.\footnote{\emph{Id}., 85}

According to this account of moral residues, the existence of the moral
residues in the above example comes from the fact that

\begin{displayquote} \def\labelenumi{(\arabic{enumi})}
	\itshape Since I promised Danny that I will deliver coal to his factory by
Thursday, Danny has a claim against me that I deliver coal to his
factory by Thursday; however, I did not deliver coal to Danny's factory
by Thursday, and thus Danny's claim has been infringed.
\end{displayquote}

\section{Problems with Thomson's Account}

However, there seem to exist several problems with Thomson's explanation
of the existence of moral residues. Suppose I promised Danny to give him
a book. However, unbeknownst to me, the book has already been burnt in a
fire before I made the promise. It seems that under such circumstances,
I would have moral residues to compensate Danny. However, it is
impossible for me to deliver the book to Danny considering the actual
empirical limitations. If I nonetheless have a duty to deliver the book
to Danny, this duty might not be trumped by other considerations and
might end up requiring me to give Danny the book. However, it is
intuitive to think that morality would not require me to do something
impossible. Therefore, morality would not give me a duty to give Danny
the book.

More generally, if an agent has a duty to do something impossible, it
might turn out that this duty is not trumped by other considerations and
would end up requiring an agent to fulfill the duty. However, it is
intuitive to think that morality would not require an agent to do
something impossible. Therefore, morality would not give one a duty to
do something impossible. In other words, if it impossible for Y to make
it the case that P, then Y does not have a duty to make it the case that
P. Since duty is the correlative of a claim, it seems that morality
would not grant X a claim against Y that P if it is impossible for Y to
make it the case that P. In the above case, morality would not grant
Danny a claim against me that I deliver him the book if it is impossible
for me to do so.

In the coal-delivery case, due to external circumstance, it is
impossible for me to deliver coal to Danny's factory. Thus, I have no
duty towards Danny to deliver coal to Danny's factory and Danny has no
claim against me that I deliver coal to his factory. Even though it
seems that Danny cannot have the kind of claim that is identified by
Thomson, there still seems to exist moral residues, specifically the
ones mentioned in part 2.

Similarly, any promises that are about something that cannot possibly be
done by an agent - let's call this kind of promises \emph{empty
promises} - do not generate the kind of claims identified by Thomson.
However, sometimes a promisor's failing to fulfill an empty promise
would still generate moral residues, as we see in the coal-delivery
example. Therefore, cases of empty promises present a problem to
Thomson's explanation of the existence of moral residues: in cases of
empty promises, the claims that Thomson identifies are implausible for
agents to have and do not contribute to the existence of moral residues.
If there is no Thomsonian claim in cases of empty promises, there is no
intermediary bridge between the non-normative fact of a broken promise
and the normative moral residues under Thomson's account.

\section{From Non-Normative to Normative}

The problem with Thomson's account makes us wonder what, if not a
Thomsonian claim, the intermediary bridge between a broken promise and
the subsequent moral residues actually is. But perhaps Thomson's
assumption that there need be an intermediary bridge is wrong and so a
better question to ask here is whether there need be an intermediary
bridge at all between a broken promise and the moral residues. Remember
that we thought that there need be such a bridge because the fact of a
broken promise is a non-normative fact whereas the fact of moral
residues is a normative fact. Would adding a claim in the middle really
help? A claim that X has against Y is the correlative of a duty that Y
has towards X, the duty being a moral reason for Y to make it the case
that a certain state of affairs obtains. It seems that a claim is simply
a placeholder for the normative fact that there is a moral reason for Y
to make it the case that a certain state of affairs obtains. By
inserting a claim in between a broken promise and the moral residues,
Thomson simply replaces the question of how to pass from the
non-normative fact of a broken promise to the normative fact of
\emph{moral residues} with the question of how to pass from the
non-normative fact of a broken promise to the normative fact of a
\emph{claim}. The question of how we get from the non-normative to the
normative is yet to be answered. Therefore, Thomson's account of moral
residues, by adding a Thomsonian claim in between a broken promise and
the subsequent moral residues, does not satisfactorily settle the issue
of how to get from a non-normative fact to a normative one.

But what would settle the issue? Surely, we need to make the jump from
the non-normative to the normative at some point. One certainty,
however, is that Thomsonian claims really cannot do this job.

In order to know why one has moral residues after breaking a promise, we
must look into what it means to make and break a promise. After all, if
I didn't promise Danny to deliver coal to his factory, or if I didn't
break that promise, there wouldn't exist any moral residue. Hence, the
explanation of moral residues must lie in what it means to make a
promise, or what it means to break a promise, or both.

\section{Promises}

Recall the concept of a promise:

\begin{displayquote} \def\labelenumi{(\arabic{enumi})}
	\itshape Y promises X that P, where P is an action or refraining from
action in the future or a limited range or states in the future, iff
	\begin{enumerate}
		\item Y invites X to rely on P's obtaining, and
		\item  X desires to receive and accept the invitation, and indeed
does so, and
		\item P has Y as its subject.\footnote{Thomson, 298-300}
	\end{enumerate}
\end{displayquote}

\noindent Promises are invitations of reliance that are accepted by the promisee
and by nature have constraining force. If promises do not have
constraining force, then the promisor\footnote{In discussing the
  counterfactual scenario that promises do not have constraining force,
  I am using ``promise'', ``promisor'', and ``promisee'' as placeholders
  for what would be promises, promisors, and promisees had the
  counterfactual been false.} may permissibly break a promise whenever
she desires. The promisor would not be doing something morally wrong if
she breaks a promise. Why, then, should the promisor be subjected to any
moral residues? Suppose I promised John to deliver widgets to him but
did not do so simply because I felt lazy. If promises have no
constraining force, then I would not be doing anything morally wrong if
I did not deliver widgets to John. I would not need to compensate John
for his financial loss. However. this picture is surely absurd - why
should John suffer financial loss due to my fault?

Furthermore, if the promisor is not bound by the ``promise'', then
rationally, no one would rely on others to carry out what they try to
promise. Why would they? Would they just hope that the promisor is
kind-hearted enough to do something that morality does not require her
to do? In other words, condition (2) of what makes a promise would never
be fulfilled - no one would accept the invitation of reliance and
consequently there would not exist any promise at all. Hence, constraint
is deeply rooted in the nature of a promise - we cannot even conceive of
a promise without the constraint carried by a promise. (What does it
mean to make a promise to me when you do not guarantee what you promised
would be obtained?) Hence, if Y promises X that P, then Y invites X to
rely on P's obtaining; and if X desires and does accept the invitation,
then Y constrains herself in a certain way.

Because of their nature as invitations of reliance, promises must have
some constraining force on people. Something that has constraining force
is something such that one has a duty to fulfill it and the failing of
which often\footnote{``Often'' is in place here because there are things
  that have constraining force but the failing of which has no
  consequence on the agent. For example, when there is no harm or loss
  associated with one's breaking a promise, supposedly there is no moral
  residues on the agent for failing to fulfill the promise.} has
consequences on the agent. The constraints of promises are thus
expressed when the promisor Y

\begin{displayquote} \def\labelenumi{(\arabic{enumi})}
	\itshape 
	\begin{enumerate}
		\item acquires a duty of making it the case that P if it is possible
for her to make it the case that P, and \\

			If Y fails to make it the case that P through no fault of X's,
then Y
		\item obtains other duties, if it is possible for her to fulfill
those duties. 
	\end{enumerate}
\end{displayquote}

\noindent Note that Y only has a duty of doing something possible for her to do. Y
only has a duty to make it the case that P if she indeed can make it the
case that P; she only has other duties as a consequence of failing to
make it the case that P if she can fulfill those duties. Throughout the
rest of the paper, I will use duty in such a way that the possibility
for the agent to fulfill the duty is implied. As you might have
expected, my explanation of the existence of moral residues does not
rely on the existence of a duty, but instead relies on what it means to
make and break a promise. Under my account, the fact that there is no
Thomsonian duty in cases of empty promises is perfectly compatible with
the fact that there exist moral residues. Hence, the fact that one
cannot have a duty that she cannot possibly fulfill would not generate
the problem of empty promises for my account of moral residues.

The locution ``through no fault of X's'' is in place in this account of
the constraint of promises. The reason is that if it is through X's own
fault that the promise cannot possibly be fulfilled, then Y has no duty
to take responsibilities for not fulfilling the promise. If Danny is the
terrorist that bombed the only factory that provides coal in the city,
then it is intuitive to think that I do not have a duty to take
responsibilities for my inability to fulfill the promise.

If it is Y's fault that he cannot keep the promise, then Y has a duty to
take responsibilities for breaking the promise. If I did not deliver
coal to John simply because I was lazy, then I have a duty to take
responsibilities for breaking the promise.

If it is due to factors that Y cannot control that the promise cannot
possibly be fulfilled, then Y has a duty to take at least partial
responsibilities for breaking the promise. After all, why should X bear
the full cost of Y's inability to keep the promise when it is no fault
of X's that Y can't keep Y's promise? Therefore, it is reasonable to
think that if Y breaks the promise through no fault of X's, then Y has a
duty to take at least partial responsibility for breaking the promise.

Barring extreme circumstances such as enforced promises, promises have
constraining force, which means that one has a duty to carry out what is
promised and that failing to do so often has consequences for the
promisor. From what I have argued above, the constraining force of
promises is expressed not only when the promisor Y acquires a duty of
making it the case that P, but also when she obtains other duties as a
consequence of failing to make it the case that P through no fault of
X's.

\section{The Constraining Nature of Promises as Intermediary}

As argued above, promises would not exist if they had no constraining
force. The constraining nature of promises can thus make the jump
between non-normative facts and normative facts. By making a promise,
one constrains herself to certain normative facts, namely that

\begin{displayquote} \def\labelenumi{(\arabic{enumi})}
	\itshape 
	\begin{enumerate}
		\item  one acquires a duty of making it the case that P if it is
possible for her to make it the case that P, and
		\item  If Y fails to make it the case that P through no fault of X's,
then Y obtains moral residues, if it is possible for her to fulfill
those duties.
	\end{enumerate}
\end{displayquote}

\noindent
Those normative facts are reasons for one to do or refrain from doing
something. We do not need any further intermediary bridge between the
non-normative and the normative.

\section{Contents of All Moral Residues}

What are the contents of the moral residues that Y obtains if Y fails to
make it the case that P through no fault of X's? To figure out the
content of the moral residues, it might be helpful to look at some
intuitive examples.

In the widget-delivery case, it is intuitive to think that I obtain
several other duties due to my failing to deliver widgets to John's
customer. First of all, I have a duty to minimize the losses incurred
due to my laziness, perhaps by delivering widgets to John's customer as
soon as possible. Secondly, I have a duty to compensate for any loss
that John suffered due to my laziness, perhaps by giving John a
proportionate amount of money. This is an intuitive case where Y fails
to fulfill a promise due to Y's fault, and X suffers loss from Y's
failing to fulfill the promise. Y obtains moral residues to reduce a
\emph{proportionate} amount of negative effect. The word `proportionate'
ensures that when it is Y's fault for breaking a promise, Y only obtains
moral residues proportionate to Y's wrong-doing. Y does not obtain
duties that are unproportionate to Y's wrong-doing. In the
widget-delivery case, just because I broke my promise to John does not
mean that I should deliver widgets to his customers for free for the
rest of my life. Of course, what it means to reduce a proportionate
amount of negative effect associated with Y's breaking a promise depends
on the particular case at hand, because different circumstances call for
different methods of remedy.

If, however, John does not suffer any loss from my failing to fulfill a
promise, then there is no proportionate amount of negative effect for
one to reduce. So, I would not be subjected to moral residues.

From the above discussion, the content of a moral residue is to reduce a
proportionate amount of negative effect. This way of defining
moral residue indicates that when there is no negative effect incurred
due to Y's breaking the promise, Y is not subjected to moral residues.

\section{Application of the Account}

Considering what we discussed about promises in parts 5, 6, and 7, the
reason that moral residues exist is thus:

\begin{displayquote} \def\labelenumi{(\arabic{enumi})}
	\itshape When Y promises X that P, Y invites X to rely on Y's making it the
case that P, and thus constrains herself in such a way that
	\begin{enumerate}
		\item Y has a duty to make it the case that P, or
		\item f Y fails to make it the case that P through no fault of X's,
then Y has moral residues to reduce a proportionate amount of negative
effect associated with Y's breaking the promise. \\
			
			And Y fails to make it the case that P.
	\end{enumerate}
\end{displayquote}
\noindent
Applying my account of moral residue to the coal-delivery example, the
reason that there exist moral residues is that:

\begin{displayquote} \def\labelenumi{(\arabic{enumi})}
 \itshape When I promise Danny that I deliver coal to Danny's factory by
Thursday, I invite Danny to rely on me, and thus constrain myself in
such a way that
	\begin{enumerate}
		\item I have a duty to deliver coal to Danny's factory by Thursday,
or
		\item If I fail to deliver coal to Danny's factory by Thursday
through no fault of Danny's, then I have moral residues to reduce a
reasonable amount of negative effect associated with my breaking the
promise. 
	\end{enumerate}
\end{displayquote}

\noindent
To reduce a proportionate amount of negative effect would include:

\begin{displayquote} \def\labelenumi{(\arabic{enumi})}
	\textsc{MR1} - I seek release from my promise to Danny that I sell him
coal \\
	\textsc{MR2} - if I do not get a release, I compensate Danny for his
financial loss.
\end{displayquote}
\noindent
First of all, I should try seeking release from Danny. If Danny says:
``don't worry about it, I have plenty of coal in my factory,'' then I am
released from my promise and consequently my duty. A promise engages two
parties, but if the promisee releases the promisor from the promise, the
promise would cease to exist.

Second of all, if I cannot be released from the promise, then I also
have a second moral residue to compensate Danny financially. On the
other hand, if I am released from the promise, then the first part of
MR2 would not obtain, and I do not have a duty to make amends.

The above two moral residues comprise my effort to reduce a
proportionate amount of negative effect associated with my breaking the
promise. They are appropriate for me to have considering my
unintentional breaking of the promise.

Under this account, even though the promisor's behavior is constrained
in such a way that she is subjected to a duty and a further conditional
duty, the duties do not explain the existence of moral residues.
Instead, it is the constraining force of a promise as an invitation of
reliance that explains the existence of moral residues.

\section{Hierarchy of Duties}

Thomson objects to a similar account of moral residues on page 94,
footnote 7: ``[this proposal makes] too much of the need to seek a
release or to compensate. For if I have promised a man to do a thing, it
is - other things being equal - not good enough that I merely compensate
him for such losses as I cause him by not doing the thing; other things
being equal, I just plain, and all simply, ought to do it."

What Thomson means in this paragraph is that among the duties associated
with one's breaking a promise, the duty to make it the case that P
should be prioritized over moral residues, since whether to do P or to
fulfill the moral residues is not up to Y. What X is relying on is not
that Y fulfill moral residues; rather, X is relying on Y to make it the
case that P. Y has a duty to try her best to make it the case that P,
before she obtains a duty to reduce a reasonable amount of negative
effect associated with her breaking the promise.

There are several things worth noticing in my account of moral residues
that are relevant to the issue of hierarchy of different duties that are
associated with one's making a promise:

\begin{enumerate}
\def\labelenumi{\alph{enumi}.}
\item
  the connective between the first and second part of the constraint
  generated by a promise is an ``and'';
\item
  the second part of the constraint is a conditional;
\item
  the order of the first and second parts of the claim;
\end{enumerate}

All three mean that if Y makes it the case that P, then Y is not
constrained by moral residues; but if Y fails to make it the case that
P, then Y is constrained by moral residues. This means that whether to
do P or to replace negative effect is not a free choice for Y. Rather, Y
has a duty to try her best to make it the case that P before she obtains
moral residues. Hence, the construction of my account of moral residues
fends off Thomson's objection regarding the prioritization of duties.

\section{Advantages of This Account}

In short, when Y promises X that P, Y invites reliance to X on the
obtaining of P, and thus constrains herself by a conjunction of a duty
and possible moral residues. Under my account, it is perfectly
reasonable for one to have moral residues without having a Thomsonian
duty. Hence, cases of empty promises do not generate a problem for my
account of moral residues.

Furthermore, according to this account of moral residues, the reason
that there exist moral residues when Y fails to make it the case that P
is that:

\begin{displayquote} \def\labelenumi{(\arabic{enumi})}
	\itshape When Y promises X that P, Y invites X to rely on Y's making it the
case that P, and thus constrains herself in such a way that
	\begin{enumerate}
		\item  Y has a duty to make it the case that P, and
		\item if Y fails to make it the case that P through no fault of X's,
then Y has moral residues to reduce a reasonable amount of negative
effect associated with Y's breaking the promise. 
	\end{enumerate}
\end{displayquote}
\noindent
The existence of moral residues, contrary to what Thomson claims, does
not rely on an intermediary of a claim to bridge the non-normative and
the normative. Instead, it relies on the nature of promises as
invitations of reliance that have constraining force, thereby avoiding
the problem with regards to empty promises under the Thomsonian account
of moral residue.

\section*{References}

\begin{itemize}[label={},itemindent=-2em,leftmargin=2em]
	\item Thomson, Judith Jarvis. \emph{The Realm of Rights}. Cambridge, Massachusetts: Harvard Univ. Press, 1990.
\end{itemize}
