\noindent Dear Reader,
\bigbreak
Like Descartes' famous work after which this journal is named, the philosophy in these pages proceeds by a spirit of criticality. As you'll discover, this manifests in the recognition that even persuasive ideas of some of the greatest minds may be wrong. In \emph{Moral Residues From Broken Promises}, Waner Zhang challenges Judith Thomson and provides an account for the origin of the moral norms generated by broken promises. John Abughattas uses Kant to understand what is the wrong imposed on the Stateless. And in \emph{The Limits of Heritage}, Aaron Peretz identifies an important shortcoming in Martin Heidegger's notion of authenticity.

Overseeing this year's production of Meditations has been a labor both rewarding and delightful. It wouldn't have been possible were it not for the dedicated hours put in by its staff. Our Selections committee spent weeks carefully reviewing a record number of submissions. Their work was furthered through the efforts of our Editors, who worked to hold this journal to the very highest standard. Hearty thanks are owed to the Undergraduate Philosophy Club, especially President Eva Yguico who was brilliant and indefatigable. A very special thanks must be given to professors John Carriero and Andrew Hsu. They always readily provided advice and direction, especially when our naivete ran into trouble.

Now in its sixth edition, \emph{Meditations} has established itself as a serious undergraduate tradition at UCLA. Just as it has grown, I have no doubt it will
continue in the coming years to blossom and bear fruit. In that same spirit of optimism, dear reader, I now welcome you to enjoy these \emph{Meditations}.
\bigbreak
\hfill
\begin{tabular}{@{}l@{}}
	Respectfully,	\\
	\scshape David Graham Dixon \\
	\textit{Editor in Chief}
\end{tabular}