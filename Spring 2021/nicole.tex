\paragraph{\small Abstract.}

{
\small
In this essay I will first present and evaluate Derek Parfit's theory of
personal identity, which says that personal identity consists in
psychological continuity, which consists of overlapping chains of
psychological connectedness. Parfit's theory leaves out an important
consideration: the transitions between one's beliefs, intentions, and
other attitudes through time significantly contribute to one's
continuous existence as a person. I will argue that personal identity
consists in the kind of psychological continuity constructed by rational
transitions between propositional attitudes. Next, I will argue that if
psychological continuity consists in rational transitions, then
psychological continuity and bodily continuity, which are traditionally
seen as alternative theories of personal identity, cannot be independent
from each other when obtaining continuous transition-prompting
perspectival experiences. \footnote{This essay is
  written for partial completion of the honors program in Philosophy at
  New York University. I could not have produced this paper without the
  help and support from Professor Paul Horwich and my peers from the
  Honors Thesis Workshop. The third section of this paper, titled
  ``Bodily Continuity and Psychological Continuity,'' comes from a
  question raised by Nate Ronnings during discussion. Most of all, I
  would like to express my endless gratitude to Professor David Velleman
  for his year-long guidance, numerous feedbacks, inspiring
  conversations, and kind support.}
}

\section{Rational Transitions }
\pretolerance=400
Derek Parfit argues that personal identity consists in psychological
continuity, which consists of overlapping chains of psychological
connectedness. A person $X$ at one time is psychologically connected to a
person $Y$ at another time if and only if $X$ has a psychological state\footnote{These
  psychological states, according to Parfit, can take various forms. For
  example, such a state may include intentions, beliefs, or a
  combination of both. The paradigm case of this kind of caused mental
  states remains to be experiential memory.} that is causally dependent
upon some earlier mental item\footnote{Mental items, here, differ from
  psychological states in this context by being specifically about some
  event. A phenomenological experience of $X$ is a mental item regarding
  $X$, whereas a psychological state of $X$ is simply relevant to $X$ by means
  of $X$ being part of the mental items held in that state.} in the same
way that an experiential memory is causally dependent upon the
experience it is about. One may think of direct psychological
connections in terms of retention of mental items. According to Parfit,
$X$ is the same person as $Y$ from an earlier time if and only if $X$ relates to $Y$ by a
sufficient amount of overlapping chains of psychological connectedness.
Overlapping chains of psychological connectedness refer to the
synchronically existing (although not necessarily synchronically
initiated and relinquished), temporally extended chains drawn out by
these causally related mental items and psychological states. Therefore,
in order for $X$ at present to be the same person as $Y$ from
three weeks ago, $X$ does not need to have experiential memories of $Y$'s
experience from three weeks ago. Instead, $X$ may have memories of some
experiences from a week ago; and a week ago, the person having the
experiences later remembered by $X$ also had memories of experiences from
two weeks ago\ldots\ So, although $X$ may not directly remember her
experiences from three weeks ago, she has overlapping chains of
psychological connectedness between her present self and that person
from three weeks ago. These overlapping chains constitute her sameness
of persons. Therefore, if a person suffers from severe amnesia, then her
personal identity is, at least, threatened if not already devastated by
the loss of overlapping chains of psychological connectedness brought
out by the holding of experimental memories.

Although Parfit argues personal identity consists in overlapping
chains of psychological connectedness, he does not think that all direct
psychological connections should be weighed in the same way since ``more
weight should be given to those connections which are distinctive, or
different in different people.''\footnote{Derek Parfit, \emph{Reasons
  and Persons} (Oxford University Press, 1984), 515. n.6.} If $X$ is
psychologically continuous with $Y$ yet no psychological connections
between $X$ and $Y$ can distinguish that person from others or convey that
person's values,\footnote{Ibid, 299.} then such psychological continuity
seems to have little significance. It fails to account for what matters
for a person's identity through time. So the psychological connections
involving the person's values are more important than other connections.
Thus, when evaluating the sameness of a person, the persistence of these
value-conveying connections should also be weighed equally as, if not
more important than, degree of psychological connectedness.\footnote{For
  similar articulation of this consequence of Parfit's view, see J.
  David Velleman ``Identification and Identity'' in \emph{Self to Self}
  (Cambridge: Cambridge University Press, 2006), 337, n.25.}

Despite its brilliance, Parfit's theory of psychological continuity
cannot account for the disturbing feeling of disassociation when
forgetting about some seemingly trivial experiences. Let us examine a
very common occasion in our daily life:

\begin{quote}
You plan to leave your apartment to get groceries. To execute this
plan, you form several intentions and act upon them in a sequential
manner: you pick up your keys and exit your door. Then you close
your door, lock it up, put your keys back into your pocket, and go
downstairs. When exiting your apartment building, you forget
whether you have locked the door. Now you find yourself standing
on the street worried and disturbed. You ask yourself: have I locked
my door?
\end{quote}
According to Parfit's view, this is a classic example of losing a
trivial piece of experiential memory without disrupting the identity
relation between the earlier and the later person. After all, the
experiential memory of locking the door does not convey your values or
distinguish you from other people. Also, when exiting your building, you
almost definitely have a sufficient amount of overlapping chains of
psychological connectedness with your door-locking self. Thus, Parfit
would not consider this piece of forgotten experience as a threat to
one's sameness of person.

Yet that is problematic. It is more than just a missing piece of
experiential memory. After all, people forget about their experiences
all the time. You would almost always forget about the faces that you have
seen on the street, but forgetting those memories from experience is not
as disturbing. So, what is so different about this particular case of
forgetting whether you have locked your door that makes it bothersome?

\pretolerance = 400
Perhaps it is disturbing because this loss of memory creates a
discontinuity in living your life. You have a goal of leaving your
apartment and going to the grocery store. You know what you necessarily
need to \emph{do} in order to achieve this goal. Then, you execute these
actions one at a time. When acting, you are in the process of
constructing a continuous sequence of action guided by practical
thoughts. This is the project you are committed to accomplish in that
period of time. Yet forgetting whether you have locked the door disrupts
you from accomplishing your project, thus interrupts you from living
your life in the way you have previously planned out, which can be seen
as a form of psychological continuity \emph{essential} for sameness of
person.

\pretolerance = 400
Therefore, I propose that personal identity consists in the kind of
psychological continuity constructed by rational transitions between
propositional attitudes. To make a transition between propositional
attitudes is to process the content of the existing set of propositional
attitudes---like reshaping existing beliefs, forming new intentions, or
discarding old plans.

A \emph{rational} transition\footnote{Rational transitions do not need
  to be conscious. A lot of rational inferences we perform are
  unconscious.} between propositional attitudes is made to meet two
types of coherence constraints: avoiding internal contradictions and
reaching means-ends coherence. Altering one's belief based on experience
is a form of rational transition made to avoid contradiction. If I have
the belief that I am now in my apartment when standing outside my
apartment, then I am prompted by my experience to replace that belief
with a new belief about my location to avoid having my belief contradict
information obtained from my experience. Forming a new intention based
on existing volitions is a rational transition done to achieve
means-ends coherence.\footnote{Michael Bratman, \emph{Intention, Plans,
  and Practical Reason} (Cambridge, MA: Harvard University Press, 1987),
  108--109.} If I have a propositional attitude that aims for a certain
end, then for the sake of coherence, I must also form the propositional
attitudes that aim for its means. If I intend to get groceries, knowing
that I have to leave my apartment to do so, then it is rational for me
to form the intention of leaving my apartment.

This kind of rational transition can account for sameness of person. A
propositional attitude can leave persisting marks on other propositional
attitudes generated by rational transitions later. When an attitude is
rationally formed or altered, it is shaped in a certain way in order to
avoid generating contradictions within the present set of propositional
attitudes and to achieve means-ends coherence within that set.
Therefore, the pre-existing attitudes are causally related to the later
formed or altered attitudes by means of shaping their contents. So, when
a later formed attitude persists, it \emph{bears the marks} of other
attitudes previously held during the time of its formation without 
requiring the previous attitudes to be maintained also. As a result,
even if some of the previously held attitudes are forgotten, they are
still, by means of marking, causally and rationally related to the
persisting attitudes that shape our actions and influence relevant
experiences. In this way, the newly formed attitudes are continuous with
earlier attitudes that have not themselves persisted. This conception of
continuity subsumes Parfit's narrow conception, given that the rational
thing to do in many cases is to maintain an attitude in the face of
incoming information when that information presents no reason to change
or replace the attitude. Thus, retention of attitudes should be
considered as the default case of a rational transition.

Now, going back to our example of forgetting that you have locked your
door, we may now better explain the feeling of disturbance caused by
this loss of memory. This experience is particularly disturbing because
the practical rational transitioning process that partly contributes to
your personal identity has been disturbed by this loss of memory. As you
plan to exit the door and go to the grocery store, you are performing a
continuous practical rational sequence consisting of rational
transitions between action, experiences, and propositional attitudes
whereas the actions planned to be executed after locking your door are
guided based on (either the fact or) the presumption that you have
locked your door. That earlier person's action (alongside knowledge of
the action having been conducted) is necessary for you to execute your
plan at the present and in the future.

Thus, by enabling prior propositional attitudes to leave
\emph{persisting} marks on newly formed attitudes, rational transitions
constitute psychological continuity.\footnote{To respond to some
  potential questions that concern this point and its relation to my
  view of personal identity, I wish to clarify: I am not saying that in
  order to be considered as persons, all our transitions must be
  rational. For my purposes, so long as there are \emph{some}
  transitions that are rational, that would suffice the requirement of
  being a persisting person.} When you are subjectively disassociated
from your earlier self by means of memory loss, your continuous sequence
of rational transitions is disrupted. In other words, this discontinuity
of practical rational sequence causes you to recognize a minor
interruption of your continuity as a person through time. Forgetting
whether you have locked the door is minor because it is only a
disruption of one rational transition process that constructs
psychological continuity with other rational transitions. You are still
the person who locked the door even though you have forgotten about it.
But this kind of interference, when multiplied, could cause a break in
your sameness of person. Thus, Parfit is right that memory loss may
disrupt personal identity. But the relation between the two is not
directly causal. Instead, memory loss may generate such an effect
because memories serve as the fundamental building blocks for rational
transitions, which then constitutes personal identity. This is why this
case of forgetting an event of your practical rational sequence is more
disturbing than forgetting the look of people's faces you have seen on
the street.

\section{Bodily Continuity and Psychological Continuity }

So far, I have argued for a kind of psychological continuity that
consists of rational transitions between propositional attitudes. In the
example of leaving your apartment, the sequence of actions is motivated
by outcomes of rational transitions prompted by egocentrically
structured information gained from experiences. The perspectival
experience of walking out of your room can sufficiently motivate you to
reach for your keys. But rational transitions do not only happen between
egocentrically structured experiences and attitudes. As a person, one
has the capacity to make the kind of practical rational transition of
acting based on objectively structured information.

In order to act on objective information, one needs to translate between
an objective self concept and its corresponding self-notion. In his
essay ``Self-notions'', John Perry defines a self notion as a repository
for action-guiding information gained via experiences. Action can only
be directly guided by egocentrically structured information. For
example, when walking to the grocery store, I can be directly guided
only by information framed using egocentric terms like ``the store is
\emph{in front of me},'' or ``turn \emph{right} and move
\emph{forward}.'' Similarly, experience delivers information that is
egocentrically structured. When approaching the store, I am experiencing
the world from my point of view. As a result, the information gained
from this experience is formulated relative to me (``there is a grocery
store on my left''). Information structured in this way can figure only
in an egocentric scheme of representation of my surroundings. Thus, a
self notion, as a repository for action-guiding information gained from
experiences, is egocentrically structured.

Perry then defines an objective self-concept as a repository of
non-perspectival, centerless information about the self. When walking to
the store, I can directly know from my perspectival experience that I am
walking forward. But I would need to translate ``forward'' into
``south'' to be able to know my objective orientation, which belongs to
my objective self-concept. Having such an objective self-concept enables
one to think of oneself as a person who exists in relation to this world
in an objective way, which includes having a physical embodiment as well
as a causal role. Conceiving oneself in such a way as being a person
attributes personhood to the subject of this conception.\footnote{In J.
  David Velleman's essay ``The Centered Self'' in \emph{Self to Self}
  (Cambridge: Cambridge University Press, 2006) 224--225, being capable
  of thinking of oneself in this way is crucial to having personhood. A
  cat may also obtain information from an egocentric scheme of
  representation and act accordingly. What it lacks (unlike a person) is
  conception of the creature obtaining this information and conducting
  these actions.}

Consider the following scenarios to distinguish between self-notion and
self-concept:


\begin{enumerate}
\item
  My friend looks at me and says: ``There is a blue sticker on your
  forehead.'' As a result, I remove the blue sticker from my forehead.
\item
  I overhear my peers' conversation and hear them say ``Nicole An has a
  blue sticker on her forehead,'' which causes me to touch my forehead
  and to remove the blue sticker.
\end{enumerate}

\pretolerance = 400

In both scenarios, I conduct the same act: namely, I remove the blue
sticker from my forehead. But in the first scenario, my action is
motivated by knowledge from my self-notion, which I obtained from the
experience of being addressed by my friend in second-personal terms. I
do not need to have an objective concept of myself in order to recognize
myself as the one addressed in that conversation. In the latter
scenario, however, my motivation is different. I am prompted by
information obtained by means of having a self-concept---namely,
knowing that information about the person whose name is ``Nicole An'' is
information about me.\footnote{Self-concept is not information attached
  to one's name. Instead, one's name is a piece of information attached
  to one's self-concept. Perry, in various papers, refers to
  self-concept as ``linking concepts about the person we happen to be.''
  However, there can be problems associated with this definition that
  shall be too lengthy to be discussed in this footnote. For Perry's own
  illustration of self-concept, see ``Selves and Self-Concept,''
  \emph{Time and Identity}, (Cambridge: MIT Press, 2010).} By
recognizing that I am Nicole An, I am motivated to touch my forehead and
remove the sticker, since I recognize that I am the subject to their
mockery conversation. This latter example also illustrates the most
basic level of action conducted based on the inter-translation between
one's self-notion and self-concept. How did this translation happen?

In order for me to act on the basis of objectively conceived
information, my practical reason must serve as an information channel
between my self-notion and my objective self-concept. If I want to use a
map to get myself to the store, I have to go through three steps of
translation. First, I need to translate my egocentrically conceived
location and orientation (``standing \emph{here}, facing forward''),
which is part of my self-notion, into information regarding my objective
location and orientation (``standing on 9th Street, facing north''),
which is part of my objective self-concept. Then, I need to translate
the objectively structured information into egocentrically formulated
instructions applicable to me. If the map shows that the store I want to
go to is 300 feet north of me, I would have to translate that
objectively formulated instruction into ``face forward and go straight
for 300 feet'' based on knowledge of my objective self-concept obtained
previously when correlating my egocentric location and orientation with
my objective location and orientation. Finally (and arguably most
importantly), in order to further direct my actions intentionally based
on my current objective self-concept, I also need to constantly update
my objective self-concept by means of translating my perspectival
experience into changes of my objective status---translating ``I have
walked forward for 300 feet'' into ``I have walked north for 300 feet,''
for example. Thus, it is by this final step of translation, my
continuous perspectival experience can be used to constantly update my
objective self-concept, enabling me to generate a continuous sequence of
action. Notably, perspectival experiences are continuous because they
are obtained by a spatio-temporally continuous body.

This final step, which enables me to carry out a continuous practical
rational sequence, depends on bodily continuity. It is within the
content of my objective self-concept that I am physically embodied in
some \emph{thing} capable of forming spatial relations with other three
dimensional objects. What's more, the continuity of that physical
embodiment enables me to have an egocentrically structured experience of
continuous movement, which is translatable, in turn, into changes in my
concept of my objective position and orientation. When walking forward,
my egocentric scheme of representation of the world is constantly
changing because my spatio-temporally continuous body, when walking, is constantly receiving new
sensory inputs. This changing scheme of representation constructs my
continuous experience, which prompts me to constantly update my
centerless conception of time and space with incoming egocentrically
structured information. By having such a continuously updated centerless
conception, I am then able to locate myself on the map during the
process and guide my future action based on that updated location. Thus,
by having a continuous body, one is able to have a continuous
perspectival experience that not only serves as the basis for conducting
action directly from experience with the aid of practical reason but
also constantly updates one's objective self-concept for producing
continuous practical rational sequences that interact with an
objectively structured world. Therefore, if one is to think of
psychological continuity in terms of rational transitions between
propositional attitudes, then psychological continuity depends on bodily
continuity.

Traditionally, psychological continuity and bodily continuity have been
presented by philosophers as alternative theories of personal identity.
But based on the discussion above, if psychological continuity is to be
thought of in terms of rational transitions, then psychological
continuity cannot be independent from bodily continuity when performing
rudimentary practical rational sequences.

\section{Applications }


In this final section, I will be responding to some thought experiments
as well as some practical applications of personal identity. But before
getting into those discussions, I would like to first examine the
significance of thought experiments. In his \emph{Reasons and Persons},
Parfit defends the importance of sci-fi style thought experiments:

\begin{quote}
This criticism [that science fiction cannot be a useful method in
providing us with what is logically required for sameness of person]
might be justified if, when considering such imagined cases, we had no
reactions. But these cases arouse in most of us strong beliefs. And
these are beliefs, not about our words, but about ourselves. By
considering these cases, we discover what we believe to be involved in
our own continued existence, or what it is that makes us now and
ourselves next year the same people. We discover our beliefs about the
nature of personal identity over time. Though our beliefs are revealed
most clearly when we consider imaginary cases, these beliefs also cover
actual cases, and our own lives.\footnote{Derek Parfit, \emph{Reasons
  and Persons} (Oxford University Press, 1984), 200.}
\end{quote}

He argues that our beliefs about our persistence are aroused by these
thought experiments. Of course, what I have given so far as an account
of personal identity can also be categorized as beliefs regarding the
subject matter of sameness of persons. But it is important to realize
that the two beliefs are different in a way that is significant to our
philosophical investigation. When we form an immediate belief when
encountering a thought experiment, we are most likely appealing to our
intuitions. But personal identity is a metaphysical problem. When
solving metaphysical problems, we are investigating the very nature of
things. Such investigation often leads us to counterintuitive
conclusions. Yet unless intuition can be a reliable indicator of truth,
we should not take it beyond its face value, especially when
investigating a topic in metaphysics.\footnote{Thanks to Professor
  Velleman for bringing up this point.} Nor should we expect a theory of
personal identity to accommodate our intuitions about ``who I was'' or
``who I will be,'' since our intuitions are based on our commonsensical
observations of persons, which should not be considered as a ruler for
evaluating metaphysical theories. In fact, some of the thought
experiments may suggest that these commonsensical observations are not
reliable by showing that sameness of persons is not always perceptible (for example, it is not observable if a person undergoes brain
transplant yet maintains the same body). Therefore, Quine is right to
point out that our intuitions about these science fiction thought
experiments cannot provide reliable indications for the nature of
sameness of persons. Instead, when thinking about personal identity in
terms of ``what it means to be a person,'' I suggest that it is more
reasonable to think about more practical situations, where the
personhood of the subject may be unclear, and how the given account of
personal identity shall be applied in those cases.

Still, it seems to be a philosophical tradition that giving an account
of personal identity involves engaging with some thought experiments. In
this section, I first apply my account to perhaps what is
considered to be the most significant thought
experiment---fission---to show that although I am not concerned with
them, rational transitions combined with the Lewisian approach of seeing
people as 4-dimensional objects can still account for sameness of
persons in peculiar cases. Then I will shift my focus to practical
applications of my account, including its application in cases of
advance directives and dementia patients, where sameness of person
becomes both controversial and crucial.

\subsubsection*{Thought Experiment 1: Brain Duplication (or Fission)} \emph{Imagine an evil scientist has caught me to conduct his
experimental surgery. He took my brain out and made a duplication of it.
The two brains are entirely identical, containing the same information
since they have the same neuronal structures. Then he put the duplicated
brain into my body and the original brain into the body of an android
which he has made prior to the surgery. Now the android and my original
body wake up around the same time next to each other. Which of them is
me?}
\vspace{1em}

According to the given account, my body is the one that my mind can
control. But which mind is mine? It is clear that my mind consists of
the set of attitudes I held before entering the surgery. But now there
are two minds sharing the same set of attitudes, yet identity is a
privilege that can only be offered to one person at one time. Although
the two identical minds will start to diverge after the surgery, when
they first regain consciousness after the surgery but have not yet
opened their eyes, the two persons occupying different spatial locations
at the same time are qualitatively identical without being
quantitatively identical. Thus, it seems that if personal identity
consists of rational transitions between propositional attitudes, then
it would be hard to determine which person---the android with my
original brain or me with the duplicated brain---would be me.


But if we are to consider Lewis's argument that persons are 4-dimensional creatures, then there already exists two different persons
before the surgery. In his ``Survival and Identity,'' Lewis claims that
in cases of fission (like the one descibed above), there are two
different 4-dimensional objects that happen to coincide for some
period of their existence, experiencing this world from a shared
perspective. If different person-stages are related to one another in
terms of being parts of the same person and that an aggregate of
person-stages, when it is contained by some other aggregate of
person-stages, cannot sufficiently constitute a person, then a person is
the maximal aggregate of person-stages.\footnote{David Lewis, ``Survival
  and Identity.'' In \emph{The Identities of Persons}, edited by Rorty
  Amélie Oksenberg. Berkeley: University of California Press, 1976. 59.} According
to Lewis's theory, before the surgery, there already existed two persons
instead of one. The pre-fission person-stages can and should constitute
the aggregates of both post-fission people. Despite having a shared
existence, they are still different 4-dimensional persons because they
are different 4-dimensional objects tracing different 4-dimensional
paths. If we are to think of persons as 3-dimensional objects, then
we may be troubled by fission cases because what is previously
acknowledged as one person is now two people. But that conclusion is
generated from a temporal perspective by processing information received
from the present and the past. Thus, if we are to see people as 4-dimensional existences, then we are justified to think that there
already exist two different people pre-fission. There is no asymmetry
generated by fission.


\subsubsection*{Thought Experiment 2: Brain in a Vat} \emph{Imagine that a human being (you
can imagine this to be yourself) has been subjected to an operation by
an evil scientist. The person's brain (your brain) has been removed from
the body and placed in a vat of nutrients which keeps the brain alive.
The nerve endings have been connected to a super-scientific computer
which causes the person whose brain it is to have the illusion that
everything is perfectly normal. There seem to be people, objects, the
sky, etc; but really all the person (you) is experiencing is the result
of electronic impulses travelling from the computer to the nerve
endings.}\footnote{Hilary Putnam, ``Brains in a Vat,'' \emph{Reason,
  Truth, and History} (Cambridge: Cambridge University Press, 1981),
  1--21.}
\vspace{1em}


This case of a brain independent from a continuous body may seem to pose
a serious threat to my previous argument that psychological continuity
depends on bodily continuity. But a brain in a vat is not truly a
counterexample to my argument, for it is an epistemological concern
instead of a metaphysical one. First, when the perception of
experiencing a world in your continuous bodily form is caused by the
computer instead of the actual experiences, you still have objective
representations of your position and orientation in mind in relation to
the world generated by the computer. If the computer is able to make you
feel that ``everything is perfectly normal,'' then likely all your prior
objective self-concepts are left undisturbed and continuously updated
after the operation.

Importantly, such a discrepancy/asymmetry only exists on an
epistemological level. On the metaphysical level, the inputs to the
brain that prompt rational transitions still depend on the continuity of
a ``body''---although it is not the body \emph{thought of} by the
mind. Instead, it is the computer that continuously sends electrical
impulses to the brain that counts as the ``body'' in this situation. The
brain is, in Shoemaker's terms, ``sensorily embodied'' in the computer.
What's more, the brain is also generating outputs (after all, the
computer is only creating virtual reality for it. It is not suppressing
its functioning). Some of these outputs are generated to aim at
producing changes in the body pictured by the mind. The computer, doing
its job, would reflect these changes through its outputs to the brain.
Thus, such an envisioned non-existing body is also volitionally embodied
by the mind. But that is not the full picture. Volitional embodiment not
only exists between the mind and its envisioned body but also the mind
and the computer. In order to do a good job deceiving the brain, the
computer needs to take these outputs into account when producing new
electrical impulses: the illusion must fit the volition of the brain. If
the brain decides to walk forward, it would send out the corresponding
neural signals from its motor cortex. To successfully deceive the brain,
the computer must produce a moving scheme of representation that
reflects a forward-moving egocentric scheme of representations. It must
have an objective representation of the world that feeds subjective
experiences to that brain. In this way, the brain is also ``volitionally
embodied'' in the computer in an unorthodox manner. Therefore, this
brain can be both psychologically continuous and bodily continuous---it
can be a persisting person.

\pretolerance = 100
Another point made by this case is the importance of the presumed
existence of an external world to our experiences as persons. 
It is
important to translate between objectively structured information and
egocentrically formulated instructions because we think there is an
objectively constructed external world existing independent from our
subjective experiences. Moreover, there must be something outside of the
mind, something like an external world even if it is not what it appears
to be to us.

\subsubsection*{Thought Experiment 3-1: Brain-Body Separation}

\emph{In his ``Where am I'', Daniel Dennett describes the experience of
having his brain separated from his body. Although his brain in Houston
is located hundreds of miles away from his body in Tulsa, it is still,
by means of advanced technology, capable of receiving inputs from that
body's sensory organs and generating output to effectively control that
body's behavior. But where is Dennett? Is he in Houston, where his brain
is, or is he in Tulsa, where his body is?}
\vspace{1em}

In order to know one's location, one would have to obtain perspectival
experiences and derive information from them. To begin with, my body is
the one from which I obtain information. If the body is to be destroyed
and a new body is provided to be controlled by my brain, despite the
possibility that I may experience certain feelings of oddness if the new
body is drastically different from the old one, I would still refer to
the new body as my body and use it to execute actions guided by my
propositional attitudes, even though many of those attitudes is formed
based on the perspectival experience obtained by the previous body.

What's more, my body is the one that can be controlled by my volition.
As discussed before, perspectival experiences are intimately connected
to actions and propositional attitudes. To be in a physically embodied
state is not just about receiving information but also involves
initiating actions by means of volition. When I intend to walk forward,
my legs are directed by that volition of mine. A body is mine if I have
ownership to that body in the way that that body can be moved by my
volition. This condition does not require my body to be able to perform
every task in the way I want (for example, despite my strong will to
jump, my back pain would prevent me from doing so successfully---I may
only come up with a hop). Instead, a body, in order for it to be
considered as my body, needs to produce volitional behavior
corresponding to my will. If in the future, a patient suffering from
locked-in syndrome can have a new body that can be moved by will, that
new body would be his because it is owned by him.

In ``Embodiment and Behavior'', Sydney Shoemaker considers the
input-receiving aspect of the person as a sign that the person is
``sensorily embodied'' in that body and the action-directing aspect as a
sign that the person is ``volitionally embodied'' in that body. He then
argues that both sensory embodiment and volitional embodiment are
criteria for being in an embodied state. Therefore, in order to say that
that body is mine (which enables me to receive perspectival experiences
that prompts my rational transitions), I would have to be both sensorily
embodied and volitionally embodied in that body. If there is a brain
that receives inputs from body $A$ and assigns actions to body $B$, then the
brain does not sufficiently own either body. What is required for
psychological continuity is the continuity of a body owned by me.

\subsubsection*{Thought Experiment 3-2: Discontinuity of the Body}


\emph{Dennett later described the incident of having his body
disconnected from his brain due to mechanical breakdown. Prior to the
breakdown, he was receiving sensory inputs from his body in Tulsa at T1.
Then his brain went into a disembodied state from T1 to a later time T2.
At T2 it was re-embodied in a new human body back at Houston, again
receiving sensory inputs. Clearly in this scenario, there is a
discontinuity of the body from T1 to T2. But was his sameness of person
discontinued simultaneously? }
\vspace{1em}

As previously discussed, psychological continuity requires the
continuity of an owned physical body. Then from T1 to T2, there are no
continuous perspectival experiences of relocating from Tulsa to Houston
generated by the traceable movement of a continuous physical body owned
by Dennett. The lack of a continuous perspectival experience would make
it impossible for him to update his objective self-concept, which would
hamper his practical thoughts and disrupt the formation, evolution, and
execution of his practical rational sequences. Recall that these
rational sequences are important constituents of one's personal
identity. Therefore, this discontinuity of the body in this case seems
to pose a threat to personal identity.

Yet such a threat is not devastating. When Dennett is re-embodied in
Houston at T2, he is holding a set of propositional attitudes that is
not only continuous (by means of rational transitions) with but also
similar to the set of attitudes he was holding at T1 (after all, there
is no new information gained to prompt rational transitions---the only
thing that prompts rational transitions from T1 to T2 would be the
\emph{absence} of incoming information). The psychological continuity
that constitutes personal identity is not devastated, therefore leaving
one's personal identity intact.

An analogy can be drawn with reincarnation. If reincarnation can truly
happen in the way that the same mind is embodied in one body at one time
and then in another body at a later time, then it is analogous to the
above scenario. It is not surprising that change of embodiment would
disrupt one's life. One would have to put up with the circumstance that
they are in a position to alter their old plans as well as some other
previously held attitudes. But just as Dennett says, such sudden change
can be adapted quickly, and it can hardly pose any serious challenge to
one's personal identity.

Think of a slightly different scenario: what if the re-embodiment is
anticipated? In Dennett's scenario, walking up in the specific
scientific institution in Houston is not a complete surprise---it is,
to some degree, anticipated. If the discontinuation of perspectival
experience is anticipated, then it would have no negative effect on
personal identity at all. If you put a medieval priest in an elevator,
he would not have anticipated the change of environment on the other
side of the elevator door. But that is clearly not the case for any
person who knows how an elevator functions. Or, if teletransportation is
to become a popular way of traveling in the future, then the
discontinuation would have little impact on one's anticipated continuity
through time. Nor would their practical rational sequences be
interrupted. The impact is marginal. After all, what matters is that I
am receiving inputs from a body that is mine. Which specific body that
would be is of little matter.

Therefore, no matter whether the person has anticipated their bodily
discontinuity beforehand, bodily discontinuity would not break personal
identity because of the persistence of my set of propositional attitudes
as well as the constraints of the embodiment relation between a mind and
a body.


\subsubsection*{Practical Application 1: Dementia}

\emph{Cathy was diagnosed with Alzheimer's Disease, which caused her to
forget her past experiences. As the disease progresses, Cathy eventually
reaches the point that she can no longer recall any of her subjective
experiences from the past as well as her past beliefs. Is Cathy, after
losing almost all of her memories, still the same person as she once
were?}
\vspace{1em}

First, it is important to note that despite forgetting her past
experiences, Cathy is still able to make rational transitions between
propositional attitudes. She may be deprived of her long-term memories,
but her short term memory, as well as her ability to implicitly reason,
are not dysfunctional. She is still able to form beliefs corresponding
to her environment and uses these temporarily existing beliefs as the
basis for her actions. Therefore, her personhood is still intact. It
would be threatened if the disease has started to inhibit her cognitive
abilities and capacities for executing rudimentary kinds of action.

Still, according to the given account, she has ceased to be the same
person as her previous self. The set of propositional attitudes
available for her does not consist of any attitudes left over from her
pre-dementia self. Thus, her current set of attitudes no longer bears
the marks of her previously held attitudes that constitute her
pre-dementia person-stages.

What's more, her future personal identity is also threatened by her
illness. With damage being done to her cerebral cortex and hippocampus,
preventing her from forming long-term memories, there is one special
kind of propositional attitude she can no longer form: long-term plans.

Plans constitute a special kind of psychological connectedness. The
point of having a plan is to have it remembered in the future, which
will motivate specific actions at later times in order to generate
certain anticipated outcomes. The specific actions, subsequently, are
motivated by intentions that are formed as sub plans. Sub plans are more
detailed and specific intentions that serve collectively as means to
achieve the end aimed by the plan. Thus, generating sub plans is a form
of rational transition made to achieve means-ends coherence. Intentions
as sub plans are often formed as conditionals---``if I have done $X$,
then do $Y$.'' The execution of a long-term plan consists of repeated
cycles of intention forming, action conducting, and information
updating. If I would like to become a lawyer in the future, then I would
have to first form the intention of getting into law school, which
requires me to form the more detailed intention of taking the LSAT. By
actually taking the LSAT and receiving my score, I can then come up with
a list of law schools I intend to apply to\dots\ Such a cycle of
forming intention to guide action and from which one obtains information
to form future intentions occurs repeatedly in order to achieve a
long-term plan like becoming a lawyer. Through the repetition of the
cycle, one directs the person who she considers to be her future self to
act based on the information obtained via actions done by the person who
she has deemed to be her past self. Therefore, by having a long-term
plan,\footnote{One's plans, especially those that are long-term, also
  carry a special weight in one's personal identity compared to other
  direct psychological connections. The plans are formed that way
  because of who they are, or more specifically, what propositional
  attitudes are in their minds at the time they form such a plan. The
  temporal set of propositional attitudes can construct a person stage,
  which not only constitutes a continuous person but also picks out a
  certain person in the world---it is, in Parfit's terms, distinctive
  to each person. Notably, plans are normative. We plan to achieve goals
  that we think it would be good to achieve. Such a plan is, in this
  way, also laden with our values and wants.

  After all, personal identity has two components: person and identity.
  Identity relation is a basic logical relation that does not
  necessarily need to be defined. When we say $A=B$, we don't really think
  about what the equal sign means. What makes personal identity a
  metaphysically interesting subject matter is because it deals with
  persons. This is what makes long-term planning a more interesting form
  of psychological connectedness than other form of direct psychological
  connectedness like experience-memory or intention-action: not only
  does it makes the person consciously aware of their psychological
  continuity and the resulting persistence of persons, it is also shaped
  by what construct a person stage in a distinctive way (like wants and
  values).} one is not only aware of but also anticipating her
persistence as a person.

Thus, besides being psychologically discontinued from her pre-dementia
self, Cathy has also lost the capacity of identifying with her past self
at any future moment. For the present Cathy, her own persistence is not
a subject matter that can be put under scrutiny. As a result, her
capacity to navigate herself (literally and metaphorically) in this
world is greatly damaged. She can no longer form the complex practical
rational sequences that construct her psychological connectedness
because of her disease. Recall the example of forgetting whether you
have locked the door. We said it was a minor disruption because it is
just one piece of memory that is missing from one practical rational
sequence. But when most, if not all, practical rational sequences that
construct one's psychological continuity are getting disturbed, then
one's personal identity is facing serious challenges. Therefore, because
of her dementia, Cathy's persistence as a person is fragmented, if not
completely devastated.

But what if, miraculously, post-dementia Cathy is able to recover from
her illness and regain her pre-dementia memories? That would seem to
pose a threat to the given account of personal identity consisting in
psychological continuity. Surely, as we just analyzed, when dementia has
set in firmly at T2, Cathy is psychologically discontinued from
pre-dementia Cathy from T1. Then when she recovers at T3, Cathy has
regained her psychological continuity with pre-dementia Cathy. So this
person is Cathy from T1 to T2 and from T3 onwards. Yet she is not Cathy
from T2 to T3. So who is she during that time period?

This may first seem to be a challenge. But when carefully examined, this
scenario is not so different from another psychological abnormality:
dissociative identity disorder (DID), also commonly known as multiple
personality disorder. Patients of DID may have multiple identities
coming in and out of existence, taking turns to control the same body.
In such a case, each identity is not psychologically continuous with one
another in the same way that two different people are not
psychologically continuous. Cathy's case is analogous to the special
situation of DID because that person from T2 to T3 (assuming that her
illness is not too severe to devastate her personhood) is just like a
different identity taking over the body that is continuous with the body
of pre-dementia Cathy. Therefore, despite its peculiarity,\footnote{Which,
  I argue, really comes from the peculiar premise that a person can
  regain all their memories after losing all of them at an earlier time.}
it is possible to have different identities coming in and out of
existence without negatively affecting the psychological continuity of
that person associated with that specific identity. This is because such
``coming in and out of existence'' happens all the time in our normal
life as we fall into and later wake up from a dreamless sleep. I will
explain this in the next section.

\subsubsection*{Practical Application 2: Coma\footnote{Thanks to Professor Horwich for
  initially bringing this up as a potential objection.}}

\emph{Let us consider Jay, who is currently in a coma. Is Jay in the
coma the same person as he was before being in the coma?}
\vspace{1em}

The answer here is intuitively obvious. We would hesitate to say that
Jay in the coma is identical to Jay before the coma. Yet according to
the given account, it seems that Jay in the coma is unable to perform
any rational transitions---there is no psychological continuity because
there is no psychological activity occurring.

However, whether Jay in the coma is identical to Jay before the coma
depends on whether he can regain consciousness in the future. I will
explain this first in terms of dreamless sleep. When you fall into a
dreamless sleep, you are not participating in any form of rational
transitions: both your sensory experiences and your conscious activities
are paused. However, when you wake up the next morning, your sensory
organs are restarted, enabling you to have perspectival experiences.
More importantly, you wake up with largely the same set of propositional
attitudes you hold before going to sleep. There is direct psychological
connectedness between the person waking up and the person going to sleep
the night before, constituting the persistence of the person throughout
that time frame.

A coma, for our purpose of discussion, can be thought analogous to a
dreamless sleep. Thus, when the person wakes up from the coma and holds
most of his pre-coma attitudes, he has persisted as the same person. But
if he wakes up with amnesia, then his degree of sameness of person may
be lowered. If he never wakes up from the coma, then since the time he
falls into the coma, he can no longer persist as a person. Different
from the dementia case, the subject in question is no longer sameness of
person but Jay's personhood. How can one be the same person if their
personhood is no longer intact?


\section{Conclusion }

To conclude, I have first argued that in Parfit's theory of personal
identity, the important consideration of practical sequences formed by
rational transitions that significantly contribute our continuity as
persons is overlooked. Then I offered an account of psychological
continuity that consists of overlapping rational transitions between
propositional attitudes, which include not only direct retention of past
mental items but also the rational alteration or formation of new
attitudes. Then I proposed that in order to have the kind of
psychological continuity I have accounted for in terms of rational
transitions, psychological continuity has to depend on having a
continuous body providing continuous perspectival experiences instead of
being independent from bodily continuity. Finally, I addressed some
thought experiments and practical scenarios where personal identity is
both controversial and crucial.

\clearpage
\section*{References}
{
\small
\begin{itemize}[label={},itemindent=-2em,leftmargin=2em]	
	\item Bratman, Michael. \emph{Intention, Plans, and Practical Reason},
Cambridge, MA: Harvard University Press, 1987.

	\item Dennett, Daniel Clement. ``Where Am I?'' In \emph{Brainstorms:
Philosophical Essays on Mind and Psychology}, 310--323. Cambridge, MA:
CogNet, 1981.


	\item Lewis, David. ``Survival and Identity.'' In \emph{The Identities of
Persons}, edited by Rorty Amélie Oksenberg. Berkeley: University of
California Press, 1976.

	\item Parfit, Derek. \emph{Reasons and Persons}, Oxford University Press,
1984.

	\item Perry, John. Self-Notions. Logos, 1990: 17--31.

	\item Putnam, H. ``Brains in a Vat.'' In \emph{Reason, Truth, and History},
1--21. Cambridge: Cambridge University Press, 1981.

	\item Shoemaker, Sydney. ``Embodiment and Behavior.'' In A. Rorty (ed.), The
Identities of Persons. Berkeley University Press, 1976.

	\item Velleman, J. David. ``The Centered Self.'' In \emph{Self to Self:
Selected Essays}, 253--283. Cambridge: Cambridge University Press, 2006.


	\item Velleman, J. David. ``Identification and Identity.'' In \emph{Self to
Self: Selected Essays}, 330--360. Cambridge: Cambridge University Press,
2006.
\end{itemize}
}



