\paragraph{\small Abstract.}
\pretolerance = 400
{
\small
Temporality is an enabler of experience. Bergson's temporal theory of
``duration'' is particularly useful for the study and writing of
narrative. Duration's homogeneity---the result of its non-spatial
awareness of time---allows for a unique experience of integration,
rather than separation, with timescape. Duration, therefore, can serve
as an accurate and meaningful medium for recording experience. This
paper identifies anthropology as a field which would benefit from the
philosophical lens of temporality. It works with a particular
ethnography, Rosalind Shaw's \emph{Memories of the Slave Trade}, which
would benefit from revision from the perspective of duration. The
ethnography would both gain accuracy and embody transcultural respect
and autonomy. Since temporality creates a medium for experience,
treating the forms of non-verbal and unconscious memory observed by Shaw
as a function of time rather than memory can integrate value into the
phenomenology of cultural realities. The method for writing with
duration involves Alfred Gell's description of $A$-series time, which has
a foundation in subjective perspective. Gell outlines how $A$-series time
enables certain lingual techniques for portraying temporality in
writing. Ultimately, philosophy of time and its language brings
authenticity, trust, and truth to representations of experience.
}

\section*{}
Between history and anthropology, writes Steven C. Caton in \emph{Yemen
Chronicle}, is a contract to most accurately represent the
past.\footnote{Steven Caton, \emph{Yemen Chronicle: An Anthropology of
  War and Mediation} (New York: Hill \& Wang, 2005).} While authenticity
in representation preoccupies many writers, a mindfulness of temporality
is foundational to this authenticity and is too often neglected. Our
sense of temporality is the medium of experience, and a loss of that
sense estranges us from the experience of experience. To approach
understanding---and further, representing---any scale of human life, we
must place our perspective in temporality. From there, we can fight to
defend the integrity and agency of each person along with each culture's
phenomenological experience.


I'd like to raise Rosalind Shaw's \emph{Memories of the Slave Trade:
Ritual and the Historical Imagination in Sierra Leone} as an ethnography
with potential to stem from temporality. Shaw's narrative depicts
cultural transformations, including spiritual relations and divination
techniques, as non-discursive forms of slave-trade memory which thrive
despite sparse and unacknowledged explicit verbal accounts. The
experience of the Temne-speaking people could be accurately illuminated
by framing slave-trade cultural influences as a form of duration rather
than memory. Since Shaw examines real people's present experiences of
the past, a temporal outlook would add proper context to representing
the Temne's reality.


As is, reading Shaw's ethnography raises questions of ownership and
agency over memory. The ethnographer's role in analysis is tenuous in
this context of uncertain proprietorship, especially as the reader asks
whether an outsider has a fair standpoint to deem local ritual part of a
greater cultural meaning. While the role of consciousness in enacting memorial techniques is perhaps unclear, her language clearly echoes
Henri Bergson's ``duration'' theory of temporality. Since temporality
creates a medium for experience, a standpoint of duration rather than
the more isolated of memory can integrate value into the phenomenology
of cultural realities.

However, there is also the question of whether duration is immediately
applicable in an anthropological study of social phenomenon. The
ontology of duration frames the Temne experience, however the theory
does not necessarily lend itself to being communicated in writing. The
work of duration can be more fully apprehended with Alfred Gell's
description of $A$-series and $B$-series time. Considering research findings
through an $A$-series lens can help a writer develop a clear use of
perspective. $A$-series time emphasizes relativity and tenseless type
statements, lending itself to describing personal experience. Writing
with linguistic ``types'' depicts distinct points of view. Delineating
these perspectives infuses truth and authentic representation into
research, analysis, and narrative.

In this paper, I will describe Henri Bergson's concept of duration, and
how it serves as an undercurrent of Shaw's research. Shaw invokes memory
when explaining the incorporation of the past into the present. I will
instead consider that integration a temporal process of duration. Then,
I will describe how the experience of duration can be relayed in writing
using $A$-series time with the employment of types and tokens. Duration in
Shaw's ethnography would express the ongoing presence of the slave trade
through the indispensable lens of temporality---specifically through
$A$-series time. The $A$-series' focus on tensed statements and types over
tokens would integrate a solidified point of view and stronger sense of
perspective, giving autonomy to the people involved.

Bergson's idea of duration grows away from concepts of temporality which
feature identical time segments. He writes that considering time as a
sequence of identical units naturally leads to spatial visualization
since homogenous units can be distinguished only by position. Without
this spatial separation, the homogenous multiplicity would merge into a
single unit---reinforcing a more unified homogeneity. Defining time as
constructed of units creates a dynamic in which even elapsed time is
treated as start and end points---another unit. Bergson in ``Time and
Free Will: An Essay on the Immediate Data of Consciousness'' supports
duration to establish and defend the integrity of ``{[}living{]} through
the intervals.''\footnote{Henri Bergson, \emph{Time and Free Will}
  (Mineola: Dover Publications, 2001), 117.}

Duration involves perceiving the past in the present: allowing the units
to blend instead of remaining alongside the other. In lieu of a
succession of sensations assuming a line, sensations in duration will
``add themselves dynamically to one another.''\footnote{Bergson, 103.}
Experiences relate to each other and indeed only occur because of the
collective influence of the past, creating what Bergson calls
``intensive magnitudes,'' each moment flowing into, and becoming, the
next (thwarting necessitation).\footnote{Bergson, 106.} These magnitudes
are intensive because each builds itself into the next: ``each increase
of stimulation is taken up into the preceding stimulations.''\footnote{Bergson,
  106} This new reality is not homogenous because of this intensiveness
which not only smudges the units but further allows them to come into
themselves, creating something completely new and unforeseeable. For
example, Bergson described the sleep-inducing rhythm of a pendulum: if
the regular oscillations lull one to sleep, the cause is not the final
oscillation, but the relation to the whole. If the recurring sensation
had remained distinct from each previous occurrence, one could
theoretically bear each oscillation, and none would be less bearable
than the last---and therefore one's sleepiness would remain stagnant.
Bergson claims that the common conception of measuring time is simply
counting simultaneities, or repetitions of identical units.\footnote{Bergson,
  108.} Only with duration is the totality truly reckoned.

Gustavus Watts Cunningham explains in his essay ``Bergson's Concept of
Duration'' that ``if time is to be thought of as real, {[}Bergson{]}
argues, the new must be ever up-springing and the forms that arise must
be essentially unforeseeable; otherwise, time is only a repetition and
not in any sense a reality.''\footnote{G. Cunningham, ``Bergson's
  Concept of Duration,'' \emph{The Philosophical Review} 21, no. 5
  (1914): 528.} This means that duration must be heterogeneous. Further,
duration describes the totality of passing time as culminating into a
frontier of temporality, coalescing into the future. This buildup
creates ``heterogeneity,'' in which the act of endurance leads into a
cumulative and constantly new sensation. Addressing space is crucial to
understanding duration. Bergson writes, ``Pure duration might well be
nothing but a succession of qualitative changes, which melt into and
permeate one another, without precise outlines, without any tendency to
externalize themselves in relation to one another.''\footnote{Bergson,
  104.} This externalization results from viewing time as a line---a
barrier to duration. It necessarily isolates the mind as ``{[}taking{]}
up a position outside {[}the line{]}, to take account of the void which
surrounds it.''\footnote{Bergson, 103.} While duration can be reconciled
with the linear and forward-moving aspects of lines, temporality no
longer can remain visualized as such because that experientially
separates the viewer from the timescape. Bergson advocates for us to
release ourselves into temporality rather than look from the outside:
``Pure duration is the form which the succession of our conscious states
assumes when our ego lets itself live, when it refrains from separating
its present state from its former states.''\footnote{Bergson, 100.}

History in the context of duration must take on a new life: the present
is a compound of the past yet completely new. There is no experience
left, finished, in the past, for the past is retained in the
present---not simply retained but given new life and form. Rosalind Shaw
illustrates the past remembered as culture rather than history. In her
book \emph{Memories of the Slave Trade} Shaw analyzes the apparent lack
of discursive memory surrounding the slave trade among the
Temne-speaking people. The slave trade, she argues, is preserved through
an unconscious influence on the people's approach to their landscape.
Shaw deems this embodied history, or \emph{habitus}, as second nature.
The memories are held, she claims, by a ``practical
consciousness,''\footnote{Rosalind Shaw, \emph{Memories of the Slave
  Trade} (Chicago: University of Chicago Press, 2002), 7.} and are
``remembered as spirits, as a menacing landscape, as images in
divination, as marriage, as witchcraft, and as postcolonial
politicians.''\footnote{Shaw, 9.} The community's past is ``forgotten as
history.''\footnote{Shaw, 9.}

Shaw's label of ``remembering'' in this non-historical world unearths
uncertainty about ownership over memory. Is Shaw justified in using the
term ``memory'' as an outsider, or is she projecting her own conscious
recognition of the slave trade onto the Temne? Interpretation pervades
the book, and readers are left grappling with whether retainment of the
past, especially physical or cultural, can verily be considered
equivalent to memory. In addition, as Caton writes in ``Henri Bergson in
Highland Yemen,'' when studying an event, one should dissolve it into a
duration rather than outline a causal or consequential ``before'' and
``after.'' Instead, one should consider the flux of states and
relationships in the context of the event. This differs from the memory
Shaw describes, which certainly exists in time but which is causally and
directly linked to the slave trade rather than a force interacting with
it.

The way Shaw describes how the past is retained despite this dissolution
of history recalls duration: \emph{habitus} employed as the past in the
present resembles Bergson's description of intensive magnitudes. Rather
than memory, the ongoing presence of the slave trade in Sierra Leone
becomes an extension of temporality. Shaw's observations closely match
Pierre Bourdieu's elucidation of \emph{habitus} in ``Social Being, Time
and the Sense of Existence'' as distinguished from memory: ``The
already-present forthcoming can be read in the present only
on the basis of a past that is itself never aimed at as such
(\emph{habitus} as incorporated acquisition being a presence of the
past---or to the past---and not memory of the past).''\footnote{Pierre
  Bourdieu, \emph{Pascalian Meditations}, (Palo Alto: Stanford
  University Press, 2000), 210.} Shaw's conceptual explanation for the
continuation of slave-trade sentiments in memory can be more fully
represented as a simulacrum and effect of duration.

Representing the Temne experience as duration drains the burden of
perspective from Shaw's rendition. That alleviates the complexity of
relaying individual subjectivity as a single author. Shifting the focus
towards temporality smoothly originates from Shaw's own language, which
models Bergson's. The present, Shaw accounts, is an ``active presence of
the whole past.''\footnote{Shaw, 5.} From there, her sense of the
present naturally slides into Bergson's.

One instance in Shaw's writing which could be reframed as temporality is
her description of how slave trade sentiments incorporated into,
therefore shifting, the ``spirit memoryscape.''\footnote{Shaw, 46.}
Before the slave trade, the Temne cherished close relationships with
spirits who brought healing and prosperity to the community. After a
shift in the spirits' nature, attributed by Shaw to the atmosphere of
the slave trade, the spirits became feared as harbingers of kidnappings
and raids. This shift physically manifested through the Temne banishing
the spirits from the town into the bush, transforming the spirit
landscape. Shaw uses the term ``spirit memoryscape'' to emphasize the
role of memory in the evolution of spirit-townsperson relationships.

Further, writing that ``{[}the Atlantic and legitimate trades{]} are
remembered through their incorporation into existing cosmological forms;
yet in the process, those forms seem to have changed beyond
recognition,'' Shaw illustrates and applies Bergson's heterogeneous
sense of time.\footnote{Shaw, 67.} Bergson establishes that the
culmination of the past completely reconstructs the present into
something brand new and therefore heterogeneous. In the case of Sierra
Leone, the slave trade is the past, and it becomes the present by
completely remodeling spirit relationships so that their nature is
unrecognizable comparatively. This is just as the present, for Bergson,
becomes both an accumulation of the past and something entirely distinct
from it. The connection is strengthened by Shaw's descriptions of
similar changes in other images and ritual forms as ``more than
anachronistic `survivals' of a past landscape. Through them the past was
embodied as an active presence in practices and perceptions of the
landscape {[}\ldots{]} that shape the present.''\footnote{Shaw, 68.} The
past does not simply continue to endure, homogenous and identical to its
existence as the present. In accordance with Bergson's duration, the
past surpasses survival by incorporating into a larger and different
present. In these examples, the rituals and relationships shaped by the
past behave as simulacra of temporality as dictated by duration.

Another opportunity for temporal application in Shaw's ethnography is in
her illustration of the Temne's layered ritual. Shaw recalls other
foreign scholars in Sierra Leone saying to her that the Temne have no
culture of their own, only borrowed. However, Temne ritual specialists
view the amalgamous nature of their practices as unique: when they
borrow, it becomes their own. Shaw observed that their ``transregional
techniques are often made to recall a local past through their reworking
by Temne diviners,'' again recalling the creation of a heterogeneous
present through a persisting past in the way that the borrowed ritual,
when realized by the Temne, is maintained in the practice but in an
entirely new fashion with differing meanings.\footnote{Shaw, 70.} Shaw
speaks to many Islamic traditions which are reshaped to reflect the
slave trade in Sierra Leone; for example, the Temne re-narrate a story
about Muhammad and Abu Bakr hiding from the Meccans into a reflection of
a similarly tense Temne landscape of Atlantic and colonial pasts. Shaw
also shares accounts of ritual practices which are borrowed from Islam
which change the intention and purpose of the action, personalizing it
for the Temne. This also can serve as a model for heterogeneity.

Regarding this borrowed culture, Shaw importantly distinguishes that
``what appears today as a pluralistic multiplicity is in fact a
historically produced sedimentation of layers of knowledge.'' Shaw
directly refers to the cultural influences as ``heterogeneous forms of
knowledge,'' just as Bergson sees temporality.\footnote{Shaw, 103.}
Culture and ritual practice becomes a representation of duration:
Bergson's past symbolized by the ritual's original tradition, and
Bergson's incorporation into and creation of the present equivalent to
the uptake and transformation of that ritual into a new interpretation.
In other words, what Shaw explains as memory can be illuminated through
heterogeneous temporality.

While duration is present in Shaw's work, in order to use this
ontological theory, a linguistic adaptation is necessary. Caton in
``Henri Bergson in Highland Yemen'' rewrites a passage of war and
mediation with duration in mind. This manifests as writing from the
perspective of someone involved in the event. On this point of view he
writes, ``Let me say at the outset that I do not claim to know what was
going through the sheikh's mind the moment he confronted the hijrah with
his accusations,'' and, ``It might help to know that every thought
attributed to him in my passage was at one time or another attributed to
him either by his followers, the mediators, or the inhabitants of the
hijrah, attempting to `explain' his position in the dispute by asking me
to put myself empathetically in his place.''\footnote{Steven Caton,
  ``Henri Bergson in Highland Yemen'' in \emph{The Ground Between:
  Anthropologists Engage Philosophy} (Durham: Duke University Press,
  2014), 251.} Ultimately, Caton explains, ``What I am attempting is a
representation of a consciousness that is the foundation of a particular
subjectivity (as Das would say), positioned in a certain way in the
social system, {[}\ldots{]} and what content and form such a
representation might take. This is a different understanding of
consciousness in a psychological sense: it is a construct or an imagined
interiority.''\footnote{Caton, ``Henri Bergson in Highland Yemen,'' 251.}

However, Caton ends his essay by ``{[}raising{]} the larger question of
whether it would be interesting and even possible to do fieldwork with a
focus on duration per se and then write an ethnography that would
capture this subjective consciousness of duration throughout the work
rather than merely periodically.''\footnote{Caton, ``Henri Bergson in
  Highland Yemen,'' 253.} In order to write a consistently temporal
ethnography, one may turn to Alfred Gell's techniques in $A$-series time.

While a parallel of duration helps meaningfully contextualize these
cultural transformations, when relaying them to others in an
ethnography, it is useful to clarify one's temporal positionality
through Gell's $A$-series and $B$-series time. The characteristics of
$A$-series time lend themselves to the question of perspective, which is
essential to the writing of ethnography. Gell's $A$-series time expresses
the subjective experience of time, while $B$-series time depicts the way
time is experienced regardless of positionality, and is therefore
equipped to represent succession over relative or changing perspectives
of time. Shaw writes ambiguously between series $A$ and $B$, using memory
instead as her framework for temporality. A conscious orientation
towards either a subjective or detached standpoint distinguishes
analysis of data from data itself. Shaw's narrative of memory and
remembering is written from an objective point of view yet speaks to the
Temne's experience. Relaying facets of A series time would allow Shaw to
develop or abandon her point of subjectivity---and to express the
duration she has written underneath her own narrative.

$A$-series time optimally represents Shaw's sense of duration because of
its outlook on the past in the present, its representation of change,
and its foundation in subjective perspective. In \emph{The Anthropology
of Time}, Gell uses George Herbert Mead's ``classic $A$-series statement''
to illustrate the incorporation of the past into the present: ``The
causal conditioning passage and the appearance of unique events
{[}\ldots{]} gives rise to the past and the future as they arise in the
present. All of the past is in the present as the conditioning nature of
passage, and all of the future arises out of the present as the unique
events that transpire.''\footnote{Alfred Gell, \emph{The Anthropology of
  Time: Cultural Constructions of Temporal Maps and Images} (Oxford:
  Berg, 2001), 155.} Similar to duration, heterogeneity in $A$-series time
lies in a reactive present---``Change results from
`becoming.'\,''\footnote{Gell, 173.} $A$-series time is ``dynamic''
(though determined, unlike duration), and comprised of an ontologically
different past, present, and future. This ontology resembles Bergson's
description that while duration endures, it need not ``forget its former
states'':
\begin{quote}
It is enough that, in recalling these states, it does not set them
alongside another, but forms both the past and the present states into
an organic whole, as happens when we recall the notes of a tune,
melting, so to speak, into one another. Might it not be said that, even
if these notes succeed one another, yet we perceive them in one another,
and that their totality may be compared to a living being whose parts,
though distinct, permeate one another just because they are so closely
connected?\footnote{Bergson, 100.}
\end{quote}
On this nature, Gell calls again to Mead, writing, ``Mead conceives of
time as a wafer-thin screen of unique events in a continuously changing
and moving present. It is presentness alone which confers reality on
anything, but the present bears within itself the residual effects of
the whole of the past, and prefigures the whole of the
future.''\footnote{Gell, 155.}

Further, Gell differentiates tensed and tenseless truth conditions,
relating them to $A$- and $B$-series time. Tensed statements depend on
context, while tenseless statements remain true despite conditions.
These statements can be applied as tokens and types, in which tokens are
true ``on certain occasions/at certain spatial co-ordinates/when uttered
by certain individuals, etc,'' and types are true ``independently of the
context of their utterances.''\footnote{Gell, 167.} $B$-series time is
founded on the tenseless truths and types behind tensed statements,
while $A$-series time recontextualizes past, present, and future as it
moves. Its relativity is housed in tensed statements and tokens, since
temporal qualities depend on placement and perspective. This further
solidifies the relationship between $A$-series time and duration, since,
as anthropologist Veena Das writes, ``\,`duration {[}\ldots{]} is not
simply one of the aspects of subjectivity---it is the very condition of
subjectivity.'\,''\footnote{Caton, ''Henri Bergson in Highland Yemen,''
  239.} Because of this subjectivity, Gell refers to the $A$-series as a
``tense'' rather than a ``time.''\footnote{Gell, 166.}

Mead's own involvement with intersubjectivity and the self points to the
relational and subjective nature of $A$-series time. Gell describes that,
``Very roughly, $A$-series temporal considerations apply in the human
sciences because agents are always embedded in a context of situation
about whose nature and evolution they entertain moment-to-moment
beliefs.''\footnote{Gell, 154.} $A$-series time becomes a ``subjective
time,'' over drawing a $B$-series hierarchy between human time
consciousness and the real nature of time.\footnote{Gell, 157.} With
this in mind, $A$-series time becomes duration based in
subjectivity---equipped to carry perspective and narrative. While the
concepts of duration and $A$-series time are not identical, a descriptive
form of duration can be informed by $A$-series.

Since types and tokens are rooted in contrasting perspectives, they
prove critical to approaching and presenting research. A clear
representation of point of view in ethnographic work can balance truth
in favor. Caton asserts that, ``The point is not to get into anyone's
head but to construct or imagine a consciousness focusing on the
perception of duration from a certain position and within a particular
event.''\footnote{Caton, ``Henri Bergson in Highland Yemen,'' 252.} This
maintains a clear standpoint within types and tokens. As an exercise in
perspective, one should keep in mind which bits of information are types
and tokens, and more specifically the where and when of each type. One
must keep straight in written work what they say versus what their
interlocutors say, along with lines between what is said, thought, and
felt.

Shaw in \emph{Memories of the Slave Trade} at times remains ambiguous
between using types and tokens, save distinct personal moments. She
enters her own head and body to describe her experiences as a woman
attempting to observe rituals reserved for males, and includes anecdotes
of speaking to diviners. In another strong moment, she writes, ``But at
first, my impression of most people's everyday memories of the slave
trading and colonial past was very similar to that of Cole\ldots''
denoting her own experience and thoughts.\footnote{Shaw, 49.}

The perspective at other times is not teased out, resting somewhere
between observing the Temne, portraying her own analysis of the Temne,
and relaying the experience of the Temne. At certain moments, she
employs the passive voice; for example, she writes, ``In the following
section, I turn to ways in which these memories are made to converge and
compete in three stories of the origin of divination,'' leaving key
concepts between her own thoughts, the thoughts of the Temne, and the
experience of both.\footnote{Shaw, 81.} Representing perspective
throughout the book would bring authenticity, trust, and truth to the
narrative.


There are, however, statements flagged with these kinds of viewpoints.
Sometimes she indicates that her ideas stem from analysis: ``The
heterogeneity of diviners' knowledge, I have suggested, is composed of
diverse layers of palimpsest memories.''\footnote{Shaw, 147.} Even so,
the drive behind the factual statements remains unclear. When she
writes, ``While some diviners recount discursive, intentional memories
of these techniques' pasts in reworked narratives of the cave hadith and
in accounts of the foreign power of `The Jinn of Musa,' they also embody
contrasting kinds of memories in their practice of divination techniques
and (more rarely) in their narration of origin stories,'' why does she
choose to take on the authority necessary to apply memory
objectively?\footnote{Shaw, 102.} A more intentional use of perspective
would strengthen the reader's understanding of the Temne's experience,
which is especially sensitive since Shaw is describing and relaying
unconscious cognitive processes.

\pretolerance = 800
Overall, the importance of perspective in ethnographic work cannot be
over-emphasized. Shaw's book can better grasp this authenticity by
engaging with temporality, which is able to become a medium for
experience. Her language already mirrors Bergson's idea of duration, and
her examples of rituals and traditions which carry memory directly
represent duration's heterogeneous nature. However, as explored in
Caton's work, when framing ethnography in duration, perspective is
central but limited. In order to support this, and continue with a
temporal lens, one may turn to Gell's $A$-series time, which aligns with
duration's heterogeneity and focuses on perspective. The intentional and
explained use of tenseless types verses tensed tokens supports
relativity and would help ethnographers maintain a distinct standpoint
within their work. This work exists in order to discover and share the
lives of others. An awareness of our own presence---and our own being in
relation to those we work to reach---is the only path we have, and
temporality is the thread through each of us which embroiders
individuals into a single creation.



\clearpage
\section*{References}
{
\small
\begin{itemize}[label={},itemindent=-2em,leftmargin=2em]	
	\item Bergson, Henri. \emph{Time and Free Will: An Essay on the Immediate Data
of Consciousness}. Mineola: Dover Publications, 2001.

	\item Bourdieu, Pierre. \emph{Pascalian Meditations}. Palo Alto: Stanford
University Press, 2000.

	\item Caton, Steven. ``Henri Bergson in Highland Yemen.'' In \emph{The Ground
Between: Anthropologists Engage Philosophy}, 234-253. Durham: Duke
University Press, 2014.

	\item Caton, Steven. \emph{Yemen Chronicle: an Anthropology of War and
Mediation}. New York: Hill and Wang, 2005.

	\item Cunningham, G. ``Bergson's Concept of Duration.'' \emph{The
Philosophical Review} 23, no. 5 (1914): 525--539.

	\item Gell, Alfred. \emph{The Anthropology of Time: Cultural Constructions of
Temporal Maps and Images}. Oxford: Berg, 1996.

	\item Shaw, Rosalind. \emph{Memories of the Slave Trade: Ritual and the
Historical Imagination in Sierra Leone}. Chicago: University of Chicago
Press, 2002.

\end{itemize}
}
