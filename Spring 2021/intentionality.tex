\pretolerance = 400
We often say of art, `It seems as though it's coming to \emph{life}
before me!' This is an aesthetic judgment---it requires a sort of
sensitivity and attention to qualities different from a normal or
descriptive attitude toward objects. I will term the aesthetic property
attributed in this judgment \emph{livingness}; to say of an object that
it comes to life before you is to attribute livingness to it. This essay
explores what specifically is happening when we say of art that it
appears living to us, with the explanation proceeding via psychological
and representational capacities of the viewer or reader. My main
argument will be that an experience of the artist's
\emph{intentionality} can play a large role---though certainly not the
only role---in having an aesthetic experience of livingness. My primary
case will be poetry as, in spite of its long and weary history with
intentionality and the use of the term in literary analysis, it is the
art form in which an author's psychology is most exposed---which is
likely why it has caused so much trouble.

To these ends, I will first explicate the terms \emph{livingness} and
\emph{intentionality}. I will address the former in the first section
and clarify precisely how livingness functions as an aesthetic concept
in Frank Sibley's sense.\footnote{Sibley, ``Aesthetic Concepts''.} The
second section will then explain intentionality and how it relates to
livingness. This will be connected to experiments in developmental
psychology which show how attributing life and intentions to objects is
a natural and fundamental psychological capacity. I will further push
this connection as it pertains to art in the contrast between what is
cliché and what is original, as it is in originality that authorial
intention becomes most evident and makes a poem seem most alive.

\section{The Aesthetic Concept of Livingness}


As stated, I use the term \emph{livingness} to mean the quality someone
is attributing when they say, in the aesthetic sense, that something
appears living to them.\footnote{It might seem that inventing the word
  \emph{livingness} just serves to unnecessarily obfuscate the matter.
  This is not the case; there really isn't another word that properly
  captures the judgment ``coming to life'' in the noun form. Words like
  \emph{lively} are too linked to ideas of joyousness or spring-like
  themes, which are not necessarily related to livingness. Indeed I
  think poetry, and art as a whole, can come to life to a subject when
  it is deeply still, eerie, morose, and so on.} It's an odd term --
rarely ever do people say `this painting has livingness'; they'll rather
say `this painting is living' or `this painting appears to be living'.
But for reasons soon to be discussed, it is important to distinguish
livingness from common conceptions of what is merely \emph{living,} as
this word can be construed in a non-aesthetic sense (namely, a
biological one)\emph{.} As such, I will use livingness when speaking of
the \emph{aesthetic} attribution of living.

Since livingness is a term used to convey an aesthetic judgment,
livingness is an \emph{aesthetic concept}. In his essay ``Aesthetic
Concepts'', Frank Sibley formulates a schematic definition of the term.
He writes an aesthetic concept is ``a word or expression such that taste
or perceptiveness is required in order to apply it\ldots''\footnote{Sibley,
  1.} I will explain more of Sibley's view below, as I think he is spot
on in his understanding. Presently however, I hope to expand this rough
outline in ways he does not, as it will help flesh out our understanding
of how aesthetic experiences and concepts function.

I take ``taste or perceptiveness'' to specify the certain way in which
we represent objects aesthetically. It is a different attitude than that
of a descriptive, or ``existential'' attitude, in the words of
Husserl.\footnote{Husserl, ``Letter to Hofmannsthal''. In one of his few
  mentions of aesthetics, Husserl compares the aesthetic attitude with
  the phenomenological attitude, in that both are concerned with how
  objects appear. This is contrasted with the ``existential'' attitude,
  which I will usually term ``descriptive'' or ``normal'' to avoid
  unnecessary confusion.} Descriptive attitudes are concerned with
\emph{what} an object is, and lead a subject to connect their
representations to some degree of actuality and objectivity in the
world. When concerned with aesthetics, however, we can roughly say the
attitude is an imaginative, interested, and perhaps pleasant or
otherwise emotionally significant approach to experiencing an object.
This attitude is less concerned with what an object is but more so how
it strikes us. Thus, I'll say the experience of livingness, i.e., `this
is coming to life!', is an experience of this aesthetic nature. For
example, you might say it of a mountain because it is poised stalwartly
on the horizon, ``greeting'' the sunrise.\footnote{Citing only pleasant
  and appreciative experiences such as these would mischaracterize the
  complexity of aesthetic experiences. For example, when things come to
  life in, say, genres of horror, such as a haunted forest or an
  uncannily life-like doll, it can be quite disturbing, but nonetheless
  striking and capture our interest. This further specifies that
  aesthetic experiences are not merely what are \emph{pleasing} to us or
  to our taste, but something that pierces us and demands attention in a
  way different than a descriptive attitude toward the world.} In this
way we personify the mountain, attributing intentions (of course, we are
not speaking of any artist's intentions in this case yet) based on how
it appears to us, making it seem as though it possesses livingness.

It is illustrative to provide an example of \emph{living} in the
non-aesthetic sense; this contrast will bring to light a number of
points that differ between the aesthetic and descriptive attitudes. The
non-aesthetic representation of living is a descriptive fact,
attributing the basic feature of living to an object due to the way it
behaves in the world.\footnote{This alludes to the experiments in
  developmental psychology that will be covered later (Simion et. al.,
  ``A predisposition for biological motion'') demonstrating the
  fundamental capacity we have in identifying objects that show
  life-like qualities, or ``biological motion'' in their words.}
Attribution of just ``living'' can be uninterested and unsurprising, or
quite surprising indeed, yet in a different sense than an aesthetic one.
For example, you might think it of a lizard because of its endogenous
motion and responsiveness to the environment---no one would find this a
surprising attribution, except for perhaps a young toddler who had never
seen a lizard. However, attributing ``living'' to something (in the
non-aesthetic sense) that is \emph{not} living can be surprising or
interesting, such as a hat that seemingly floats of its own accord. In
this case it would be incorrect to believe and say `this is living';
rather, we would say `this \emph{appears} to be living'. Yet this is
\emph{still} different from saying `this has livingness' or,
equivalently, `this appears living \emph{in the aesthetic sense}.'

An example that teases this apart fully is as follows. Imagine that two
viewers behold a painting in a museum. One comments, `Isn't it amazing
how it just comes to life before you?' This is clearly an aesthetic
judgment attributing the term livingness, which could be explained in
reference to some characteristics in the painting. Before the other can
reply however, at just that moment the painting actually begins to slide
along the wall in a zig-zagging motion seemingly by its own volition --
let's say because of some contraption an impish museum curator
installed. At this moment the viewers might judge that the painting
certainly \emph{appears} as though it is coming to life (at least for a
few fractions of a second), but they do this in the non-aesthetic,
literal sense. This may be surprising to the viewers, yet it is still
unlikely they would believe the painting really is alive. They would
search for a cause, such as the contraption moving it.

The two judgments differ in ways that are illustrative of the difference
between the aesthetic and non-aesthetic. The first way they differ is
that they point to different facts about the object. One picks out
certain characteristics in the work that are of aesthetic interest to
the viewer; the other picks out the spontaneous motion of the painting.
Second, the aesthetic attribution of livingness does not require any
sort of belief in how the thing actually is---indeed you suspend
disbelief to talk about such judgments. Yet the case of the painting
literally moving requires the viewers to search for an explanation as to
why the painting is not \emph{actually} living. Further, while the
aesthetic case is appreciative and delightful because it contradicts
expectations, the non-aesthetic case would be upsetting and disturbing
for the same reason! Thus, there are meaningful and distinct ways in
which the attribution of living might be used in the aesthetic or
non-aesthetic sense.

You'll note that this exposition has often pointed to the fact that we
attribute livingness \emph{because} \emph{of} some perceived or
represented characteristics. Indeed, when Sibley speaks of aesthetic
concepts requiring ``taste and perceptiveness'' it is necessary that
there are some characteristics of the object that provide the basis of
this exercised taste or perception. This prompts the question: on what
basis or in reference to what characteristics do we employ the concept
of livingness\emph{,} or all aesthetic concepts for that matter? What
qualities contribute to or cause such experiences? It seems fairly
obvious how we attribute living---we judge based on its capacity for
internal motion, whether it grows, if it reacts to its environment, etc.
It's clear given any object (with the exception of the odd case here and
there, like a virus) that we could determine it living or non-living.
This is not so when determining livingness.

To return to the works of Sibley, this is not an easy issue to resolve.
Sibley argues that aesthetic concepts are hardly ``condition governed'',
that is, they are not the sort of concepts that are decided by necessary
or sufficient conditions.\footnote{Sibley, 4.} There is no formulation
in which one can say `If $X$ possesses the aesthetic concept $Y$, it will
necessarily follow that it has characteristics $A, B, C$, and so on' nor
is there a formulation `If $A, B, C$, etc., are present, then $X$ possesses
the aesthetic concept $Y$'. Therefore, we do not have a complete catalog
of characteristics that we can readily point to such that we decide the
case that something has livingness (or is joyous, or tortured, or warm,
and so on). Indeed, characteristics that might be said to commonly lead
one to judge an object to have some aesthetic quality can deny that
judgment in another object. A pertinent example Sibley provides is when
he writes, ``One poem has strength and power because of the regularity
of its metre and rhyme; another is monotonous and lacks drive and
strength because of its regular metre and rhyme.''\footnote{Sibley, 7.}

Nonetheless, we can say that there are some characteristics that can
\emph{likely} connect to an aesthetic concept. For example, orange and
red shades found in sunsets are likely to characterize a painting as
warm. To return to livingness and poetry, I argue that an
author's intentionality, especially when exercised in an original and
novel way, often counts for and not against livingness in the Sibleyan
sense. You should always keep in mind, when it is not explicit, that
this is no guarantee. Indeed, if an author intends to write a poem that
is entirely formulaic or unvarying from the cliché, or a poem that has
no adherence to structure and poetic conventions, it is very likely the
poem will seem \emph{lifeless} or too disorganized to be a coherent,
living thing. Intentionality is no decisive factor that a poem possesses
livingness. Before I draw these connections, however, it is time to
clarify precisely what I mean by intentionality in this context.

\section{Representations of Intentionality}

When speaking of a reader's representations of an author's
intentionality, I will have four distinct notions in mind. I believe a
reader can represent all four notions of intentionality when reading a
work (or viewing other forms of art), sometimes stratified, sometimes
blended together. When reading, we can certainly form ideas about 
	\begin{enumerate}
		\item the
general mental contents of the author, either presented in the work or
underlying it; 
		\item what the author \emph{meant} while writing; 
		\item their goals to communicate their meaning; and 
		\item what moods and
phenomenological states they underwent while writing, what writing it
was \emph{like}.\footnote{This use of phenomenology is in a fairly
  standard sense: ``Phenomenology is the study of structures of
  consciousness as experienced from the first-person point of view''
  (Smith, ``Phenomenology'').}
  \end{enumerate}
We can state these representations as
unspecified descriptive facts about the work. For example, we might say
or think `the author had some mental contents while writing this' (this
is in the first sense) or `the author had some conscious experience of
these mental contents in a certain way while writing this' (in the
fourth, phenomenological sense). When we do this, we are certainly
pointing out objective facts relating to the work.\footnote{There are of
  course fringe cases in which we would be \emph{mistaken} to do so,
  such as if a robot randomly generated a poem, or some ant accidentally
  tracing a haiku in the sand. I am not so interested in these
  complications; I am dealing with typical poetry and artwork in which
  intentionality is some constitutive feature of the work, i.e., it was
  made by someone.} However, when we interpret the intentionality or
regard it in an aesthetic attitude---that is, approach it in an
imaginative and curious manner, interested in how it \emph{seems} to us
-- this is when we represent intentionality as an aesthetic
characteristic. For example, if we were to say, `the author must have
been reminiscing of his mother' or `it is clear through his tone that he
voiced this with deep passion and pain', we place our own interpretation
on the latent intentionality in the work and form aesthetic judgments on
these grounds.

I believe it is these interpretive representations of intentionality
that contribute, in some cases, to an aesthetic experience of
livingness\emph{.} When a reader represents intentionality, especially
in the phenomenological sense, they may feel as though the author is
coming to life through their work. This can come about in many different
ways, of which I will only discuss a few to provide sufficient examples.

The first is when one connects with the emotions they believe are
portrayed in the poem (through the speaker, the tone, the diction, etc.)
and attributes them to the author's intentionality. In such a way they
might be attributing all four uses of intentionality above. The reader
can imagine the following beliefs distinctly or in conjunction: 
	\begin{enumerate}
		\item the
author was feeling \emph{x} or thinking of \emph{x} while writing;
		\item the author meant to communicate they were feeling \emph{x} or thinking
of \emph{x};
		\item the author conveyed \emph{x} to make us feel \emph{y} or
to make the poem seem \emph{z};
		\item writing of \emph{x} must have been
like or felt like experience \emph{y}.\footnote{It is worth specifying
  here and later that the reader doesn't need to be correct about their
  representations of the author's intentionality to have an aesthetic
  experience. Indeed the author could have been feeling joy or relief
  when voicing something tragic---or they could have felt indifferent
  and smug, just hoping to bag some money through their flowery words.
  Whether or not a reader ``gets it right'' in their interpretation has
  no bearing on their appreciation of a work. To further complicate the
  matter, the reader could even imagine the author meant to have such
  insincerity, that the author is cleverly disguising their emotions or
  speaking through some character's voice for dramatic effect. These
  representations can also enhance or detract from the aesthetic
  experience of livingness in a poem.}
	\end{enumerate}
These beliefs might lead the
reader to regard the author as a friend, a confidant, a suffering
companion, and so on. They are moved to empathize or sympathize with
these emotions or thoughts they believe the author had. As a result, the
attributed livingness in the poem is due to the reader having a genuine
and organic experience with a seemingly living entity, communicating
with them as they would with the people around them.

In a second case, a reader can be impressed by the brilliance or
expertise of the poet, making judgments such as `the way in which they
describe the scene makes it come to life before me' or `their cutting
language pierces through me'. In these ways, we might say they are
representing the author's intentions to \emph{be} understood and to be
seen as an artist. By fulfilling this imagined desire of the author, the
reader has a pleasant or emotional experience in this mutual
communication and understanding. In these two cases, the intentionality
is explicitly and immediately represented, becoming a primary
contributing factor to the experience of livingness. It is an experience
more common in confessional and Romantic poetry, in which the author has
full expressive power and is received as such by their audience.

\pretolerance = 100
But what of poems that don't put the spotlight on the poet? This more
complex example is when we experience livingness and it is not
\emph{immediately} in virtue of the author's intentionality. In this
case there are other elements in the poem that appear living; it seems
most common when the poem surprises us and deviates from a cliché that
it springs to life. I argue this surprise can lead us to \emph{search}
for what caused the poem to jolt us so, which leads us to the author's
intentionality as the ``animating force'', so to speak. It leads us to
puzzle over what they meant, what they were thinking as they wrote, and
whether they meant to surprise and delight us. This phenomenon will
require more explanation, and I think it will be best to draw a
comparison once again between attributing living in the non-aesthetic
sense and livingness. This comparison will be centered on studies in
developmental psychology, particularly the work of Simion et.
al.\footnote{Simion et al., ``A predisposition for biological motion''.}
and Luo \& Baillargeon.\footnote{Luo and Baillargeon ``Can a
  self-propelled box''.}

Simion et. al. have determined that infants attribute what they call
\emph{biological motion} as early as two days old (likely to be innate)
with very minimal conditions.\footnote{See note 11.} When shown animated
dots on a screen in the shape of moving animals (in this case, a dotted
outline of a walking chicken), the infants show a preference and
interest in what appears to have biological motion over displays that
show random motion not organized in animal shapes.\footnote{This is of
  course to say they show preference and interest in the
  \emph{descriptive,} not \emph{aesthetic} sense. While we usually talk
  of preference and interest in an aesthetic sense, it's understandable
  that an infant would have interest in a descriptive attitude because
  everything is novel and surprising to them.} It of course would be too
far to say the infants ``represent'' something as complex a concept as
``living'' at this stage. Nonetheless, this example demonstrates just
how fundamental and natural it is for us to see something as living and
take interest in it.

Now we can take this further and see how biological motion figures into
\emph{goal attribution}. In Luo and Baillargeon's study, they found that
infants as young as 5 months old expected a self-propelled box to have
``preference'' for the object it was shown to move toward and touch
versus an object that the box had not moved toward.\footnote{See note
  12.} They would be surprised if, when the objects' positions were
switched, the box moved toward and touched the object it hadn't been
previously seeking. Again, it is too far to say the infants develop
representations of intentionality from this. Nonetheless, we can reason
that the infants attribute some sort of rough goal to the
self-propelling box, that the box ``prefers'' or seeks out an object. We
can use this to explain how we detect intentions and goal-setting in the
non-aesthetic sense. It is very natural as adults, even when we know
better, to imagine an object to have life and intentions when it
exhibits some sort of endogenous motion and consistent action.

Here's how we can connect this to aesthetic experiences in poetry. As
explained above, one of the simplest criteria for representing something
as living (and further, attributing intention) is that it can exhibit
biological motion, i.e., self-propel, move without outside forces or
even act against outside forces. Let us then equate common poetic
structure, such as a sonnet, with regular, non-organic motion, like a
ball rolling down a ramp. Imagine an author who takes the structure of
the sonnet and then, with an utter lack of inspiration, formulaically
plugs in words to compose the sonnet. They make trite rhymes such as
`fly' and `high', `love' and `dove'. They write about an over-worn
theme, such as their beloved. Their volta is awfully bland, predictably
lamenting over how their love was spurned. In no way do they vary from
the sonnet tradition, written and rewritten for hundreds of years. It's
very likely that to one who has read a few sonnets, this poem will seem
to be dull and \emph{lifeless;} it will fail the conditions needed to
have the aesthetic quality of livingness. Rather, it will fall with an
uninteresting thud, like a rock dropped from a balcony. There is no
variation in the law-like motion. Now, if a rock were to turn around and
vengefully return to the one who dropped it, scoring a nice bruise on
their brow, this certainly would be cause for surprise. We might imagine
that the rock was \emph{living,} and had a rather spiteful goal or
intention.

In this sense, a poet who varies from the scheme of a poetic genre does
exactly this. In creative strokes they intervene in the predictable
motion of the poem, interjecting slant or off-rhymes, varying the
syllables or spacing, enjambing lines or breaking them abruptly. These
variations surprise the reader, and stand out as distinct experiences,
notably differing from the ``non-biological'' motion of unoriginal
poetry. The poem seems as though it comes to life in this way. This
surprise prompts the reader to think about what \emph{causes} such
surprise, which can lead them to the author.

An artist's intentionality is most noticeable to the reader when they
break clichés or norms. A reader can choose (or intuitively be led) to
represent these moments of intentionality in the aesthetic and
interpretive sense. They engage with and playfully imagine what the
author's psychological state might have been when introducing these
unique and idiosyncratic moments. The reader may feel \emph{as though}
they're peering into the head of the author. As such, they have an
aesthetic experience of livingness, constituted by two experiences. The
first is the initial surprise when the poem varies from what the reader
expected, making it seem ``alive''. This surprise can lead one to
attribute this to the author, at which point they represent and engage
with their intentionality as though they were still living.

With all the pieces set, we are now ready to see this aesthetic
experience in practice. I will examine two analyses of sonnets, both of
which are centered on the author's use of enjambment to break cliché
sonnet structure. Enjambment is when, in a poem, two separate lines run
together without pause, such as if it lacks a comma or the clause is
only completed by reading on to the following line. My goal in citing
these analyses will show how a reader can notice the specific techniques
an author uses to break clichés and surprise them, prompting the reader
to reflect on the author's intentionality. I will further argue that the
reader's capacity to do so enables them to judge the poem on the grounds
of the aesthetic concept of livingness, and that this phenomenon
contributes to their aesthetic experience of the poem as a whole.

\section{Two Cases of Intentionality Experienced Through Enjambment}

In Hobbs' book \emph{Literature and Cognition,} he applies coherence
theory and discourse analysis used in cognitive science and AI research
to the understanding of literature. He analyzes ``Sonnet 20'' by John
Milton, often referred to as ``Lawrence of virtuous father virtuous
son''. His review is quite thorough and splits each line into two or
three segments; as such, I will only deal with his analysis of two
lines, 5 and 6, that will suffice to illustrate that paying attention to
the intention of the author does yield fruitful discussion and aesthetic
representation. His focus is to use these lines to argue for an
interpretation of the meaning of the poem. My goal is to use his
insights to demonstrate that such sensitivity to the intention of the
author contributes to an \emph{aesthetic} experience. The two lines in
question are 5 and 6, and I've included the first four to give context.
I also recommend settling in to a space of receptivity and thoughtfully
engaging with the poem---the attitude of reading philosophy can often
detract from an aesthetic one:

\begin{quote}
{\parskip = 0em
Lawrence of virtuous father virtuous son,

\hspace{1em} Now that the fields are dank, and the ways are mire,

\hspace{1em} Where shall we sometimes meet, and by the fire

\hspace{1em} Help waste a sullen day, what may be won

From the hard season gaining; time will run

\hspace{1em} On smoother, till Favonius re-inspire
}
\end{quote}

The first four lines of the poem provide the setting and characters: the
speaker and the one he addresses are facing a harsh winter (the ``hard
season gaining'') and sometimes meet by a fire to spend leisurely time
together. It seems that the semicolon in line 5 above concludes the
discussion and establishment of time, place, and the meeting of the
characters. In this context, the statement ``time will run'' is a stoic
and bleak statement about the season; Hobbs writes, ``The reader's first
impression is that the same idea is being repeated and
emphasized.''\footnote{Hobbs, 121.} However, once they proceed to line
six the reader realizes that line five and six are enjambed, and
withholding a comma until ``smoother'' suggests the complete clause is
``time will run on smoother''. This changes the interpretation to the
exact opposite of what one first expects. The statement ``time will run
on smoother'' is a wishful and expectant statement rather than a bitter
one. The characters hopefully await until Favonius (a Roman god of
spring) reinspires the earth.\footnote{Hobbs, 120-121.}

This change in the interpretation surprises a reader. By varying from
the traditional sonnet structure in which each line pauses at the end
and contains a complete clause (see lines 1 and 2 as an example), Milton
lulls us into believing that this poem will follow standard, law-like
sonnet structure. However, he anticipates this and plays off of our
expectations. He varies the lines and makes the poem seemingly ``jump''
out of this uninteresting progression, analogous to something with
endogenous, biological motion. This captures our interest, as we are
naturally predisposed to take interest in things that appear living.

It is clear that what underlies the reader's shift in interpretation is
their grappling with what the \emph{author} meant. One thinks Milton's
meaning is to reinforce the theme of the poem, but he really intends to
introduce the turning point with the enjambment from lines 5 to 6. We
might ponder over what he was thinking and feeling when writing this, if
he hoped a clever reader would notice and delight in this move, if he
anticipated our surprise at this moment. This is where the aesthetic
experience of livingness is reinforced in the context of intentionality
and we pleasurably connect with the poem in a distinct way. By
representing this intentionality, we might feel as though we share in
communication with Milton, like we understand some private, clever
utterance he shared with us. Hobbs sums up this sentiment in one of the
very last statements in his book: ``Then what distinguishes poetic
discourse is not so much the shape of the work that the writer executes.
Rather, it is the special relationship he establishes with his reader,
demanding the best of both writer and reader, communicating important
insights, and demonstrating the depth to which we are
understood.''\footnote{Hobbs, 171.}

Let us examine another case then, this one presented in Timothy Steele's
\emph{All the Fun's In How You Say a Thing.} This will focus less on the
surprising aspects of a poem, but on a poet's use of enjambment to match
up with the sensory qualities of the poem. This reinforces the analogy
between the perceptual and physical attribution of \emph{living} and the
metaphorical representation of livingness\emph{.} The poem in question
is Browning's ``My Last Duchess''. Steele writes, ``In most instances,
he {[}Browning{]} uses the run-ons {[}enjambment{]} to suggest
rhythmically the thing he is describing. For example, when he writes

\begin{quote}
The bough of cherries some officious fool\\
Broke in the orchard for her\ldots{}
\end{quote}
the enjambment---the breaking of the pentameter over into the next
verse---serves as a rhythmical analogue to the activity of the
branch-snapper.''\footnote{Steele, 99.} When we read such descriptions
in a poem, indeed we imagine the scene before us and engage in sensory
representation. This may be enough to constitute some experience of a
scene playing in a life-like manner before us. However, such
representations are enhanced and focused by literary techniques such as
the one described. Poems do not just convey scenes through words; poets
control the spacing, meter, and punctuation in ways unique to the art
medium such that they parallel what is being said. When a reader
represents the poet's intentionality such that the poet purposefully
employs these techniques, there's an additional layer of livingness
appreciated. The reader might feel as though they're in private and
subtle communication with another's brilliant mind, working with the
author to construct meaning.

Steele presses on and cites a few more lines in which Browning employs
this technique. This analysis centers on the phenomenological
representation of intentionality. He quotes the lines:
\begin{quote}
\ldots all and each\\
Would draw from her alike\ldots{}\\
and when he refers to the calling of\\
\ldots that spot\\
Of joy into the Duchess' cheek\ldots
\end{quote}
and follows it with commentary: ``the enjambments convey the Duchess's
warm responsiveness to---and her spontaneous and candid delight in --
the world around her. Reading these lines and reading through their
turns, we may remember getting off a plane or a bus and catching sight
of someone we loved whose eyes lit up and whose face blushed happily to
see us.''\footnote{Steele, 100.} We have representations of
intentionality in many layers in this analysis. First, the most
immediate representation, and the one provided by Steele is of the
\emph{character's} phenomenal states---the Duchess. Her mental states
are being directed in such a way as being joyous, ``warm'', delightfully
regarding her environment. We enter into an empathetic state and share
in the Duchess's way of seeing the world, leading to an aesthetic
experience of not only, say, joyousness\emph{,} but of livingness
\emph{as well,} in that we feel we are relating to a character coming to
life before us.

However, this intentionality is not of the author's; it is of the
fictional character or the poem's contents. We connect with Browning's
phenomenal states when we represent him thinking of such a beautiful
sight---perhaps he thinks of his wife, or a love from his youth. We
feel his excitement and heart ache, sharing again in an empathetic state
in which we feel as though we connect with the author in mutual
understanding. Finally, we have a representation of intentionality in
the second and third sense (as described in Section II), as in Hobbs'
example above. We take notice of Browning's enjambment and see how he
ran these lines together to convey this passionate regard for the world
expressed by both the Duchess and the speaker. We can delight in this
manipulation and appreciate it, producing yet another representation of
livingness derived from the author's intentionality.

\section{Some Counterarguments and Their Responses}

In what has been said above, I have shown that livingness is a specific
aesthetic quality that captures the sentiment of when something comes to
life (in a non-literal, aesthetic sense) to someone. I have further
argued how intentionality ties into livingness, particularly using
examples in which the author's intentionality serves as an ``animating
force'' so to speak; their manipulation of a reader's expectations
surprise us and impress us, making the poem seem as though it is a
living, acting entity. I illustrated this with analogies to biological
motion and attributing intentionality to non-living, moving objects,
which seems parallel to attributing intentions to art that surprises us
and varies from law-like, inorganic, cliché motion.

In the examples I have given, one could certainly point out that the
author's intentionality does not \emph{need} to be represented at all to
have these experiences of livingness\emph{.} A few scenarios could show
situations in which this is the case. A reader may skim through the poem
and not take notice of any of these technical literary elements. They
could be, at worst, disinterested in the poem and have no aesthetic
experience. In a slightly better situation, the reader takes note that
the poem indeed felt a little ``alive'' to them, but can't put their
finger on exactly why. With a yet more sophisticated experience, the
reader might have a reaction to the livingness and cite qualities that
are \emph{not} attributed to the author's intentionality, such as the
life-like and realistic descriptions, or strictly the intentionality of
the characters in the poem. It is true that aesthetic experiences are
often immediate and emotional, and to accurately describe what is
happening when we have such experiences, it may be unnecessary to
reference these sorts of intellectual and interpretive filters.

I do not need to refute the possibility of these counterexamples to
still conclude what I have thus far. The instances of other aesthetic
experiences do not deny our capacity to render the particular judgment I
have explicated here. Nonetheless, running through each of these other
cases might elucidate some clarifying points. The first case in which
someone is entirely unimpressed does not require much discussion. As
Sibley writes, ``\dots a man need not be stupid or have poor eyesight to
fail to see that something is graceful. Thus taste or sensitivity is
somewhat more rare than certain other human capacities; people who
exhibit a sensitivity both wide-ranging and refined are a
minority.''\footnote{Sibley, 3.} This is not to commit to some elitism---Sibley argues that most people employ aesthetic concepts in day to
day life. However, it does require some experience and knowledge to
point to what makes a poem good or bad, and further I believe it
requires the right disposition. If one thinks poetry is a waste of time
altogether, they will be ruling themselves out of such experiences. If
someone is tired, hungry, stressed, has been reading for too many hours,
etc., then regardless of their aesthetic sensitivity, they may miss a
nuanced judgment they would have otherwise had.

I can treat the second and third counterexamples at the same time. My
first point of defense is that if one does not reference the author's
intentionality but nonetheless has an aesthetic experience, this in no
way denies their capacity (or anyone's capacity) that they \emph{can}
represent authorial intentionality. Someone who does not make this
connection in one poem might see intentionality as a contributing
characteristic in another, or when revisiting the same work come across
this discovery.

My second point is that even if someone does not know why they feel as
though a poem is alive, or if they cite characteristics other than
intentionality, this does not exclude the possibility that authorial
intentionality was an \emph{implicit} factor in their judgment. Indeed,
a critic might explain to this reader why they believe the author's
intentions are what animate the poem, and the reader could say, `Yes!
That's exactly it' or `I hadn't thought about that, but when you put it
that way, I certainly agree.' It would be too strong for me to argue
that therefore authorial intentionality \emph{always} implicitly
underlies these sorts of judgments---this is not the case. We must keep
in mind Sibley's argument that there is little that is logically
guaranteed or necessitated in aesthetics. There is nothing wrong with
having aesthetic experiences of livingness that do not involve
intentionality. Nonetheless I have made a case for why and how we can
have these aesthetic experiences, and I further hold that these
judgments are not uncommon due to our fundamental predispositions to
personify and imagine things to be alive.

Now, if you take little interest in the world of aesthetics or
literature, this may seem like an unimportant or modest claim. So I have
established what it's like to have an experience of livingness,
the way in which we represent such experiences, and how intentionality
can play a major role in contributing to this phenomenon. What of it? I
have three responses to this cynicism. The first is that this is a move
in arguing, in small steps, for a valid place for intentionality in
aesthetic experience. Prejudice against intentionality is still alive,
usually in pop or overzealous interpretations of ``The Intentional
Fallacy'' or other New Criticism era thought.\footnote{Wimsatt and
  Beardsley, ``The Intentional Fallacy''.} However, my argument lives
comfortably within the conceding points of anti-intentionalists, of whom
Lamarque writes, ``Indeed Wimsatt and Beardsley readily admit that an
author's `designing intellect' might be `the \emph{cause} of a poem';
they deny only that it is a \emph{standard} for judging the poem. Also
they are happy to acknowledge intentions \emph{realized} in a
work.''\footnote{Lamarque, \emph{The Philosophy of Literature,} 119.} My
thesis is a specification of \emph{how} intentions are realized in a
work, and the aesthetic impacts they have. As such, I see this as
elaborating on the common ground and showing how intentionality can
function in interpretation and aesthetic experience even in a post New
Criticism era. Further, I do not argue that this use of intentionality
is a ``standard'' for judging a work. I am merely pointing to the fact
that we \emph{do} have such representations, and that these \emph{can}
figure into our appreciation of the work. It may be that one wishes to
exclude these sentiments in formal critique, that citing intentionality
when analyzing a poem will lead one astray due to subjectivity and the
elusiveness of ``really'' knowing an author's intentionality. This is
fine in those contexts, but when our goal is to describe how experiences
of livingness (and other aesthetic qualities) work, it is disingenuous
to count intentionality out.

My second claim is that this is a model exercise in approaching
aesthetics. Despite our strivings, the realm of aesthetics is not yet a
mastered frontier, let alone a coherent and well-explained sphere of
human thought. It is hard to get specific about how we make aesthetic
judgments and what their nature is. We should not appeal to universal
principles as the classical, medieval, and early modern tradition did.
Historically, this gets us more turned around than clear about the
matter. We ought to provide examples that can be understood and are
common to our experiences---building a bottom-up theory of aesthetics.
This essay is the long version of one such example.

Finally, my third response is less philosophically established, but a
belief I hold dear to my heart. Paying attention to how we personify and
have a unique capacity for imagining and sharing meaning with others
promotes an appreciation and recognition of beauty in the world. As
Hobbs writes, ``Much of what is most powerful in literature is a
conjunction of the two categories---the fictional narrative. It is an
author's invitation to the readers to a mutual imagining, to delight and
instruct, by the creation of a possible world and possible characters
striving toward goals, told in a way that directly reflects our own
experience as we plan our way toward our goals in a world that denies us
so much of what we desire.''\footnote{Hobbs, 40.} Explaining how
intentionality functions to deliver livingness can give one a heightened
sensitivity in employing this capacity. So I invite you, mutually
imagine and delight in the intentions of others and the world around
you, fictional or not. It may just serve to increase your appreciation
for what surrounds you.


\clearpage
\section*{References}
{
\small
\begin{itemize}[label={},itemindent=-2em,leftmargin=2em]	
\item Hobbs, Jerry. \emph{Literature and Cognition}. CSLI, 1990.

\item Husserl, Edmund. Letter to Hugo von Hofmannsthal. ``Letter to
Hoffmansthal.'' Göttingen: Göttingen, January 12, 1907.

\item Jacob, Pierre. ``Intentionality.'' Stanford Encyclopedia of Philosophy.
Stanford University, February 8, 2019.

\item Lamarque, Peter. \emph{The Philosophy of Literature}. Blackwell, 2009.

\item Luo, Y., and R. Baillargeon. ``Can a Self-Propelled Box Have a Goal?:
Psychological Reasoning in 5-Month-Old Infants.'' \emph{Psychological
Science} 16, no. 8 (2005): 601--8.

\item Sibley, Frank. ``Aesthetic Concepts.'' Essay. In \emph{Approach to
Aesthetics: Collected Papers on Philosophical Aesthetics}, edited by
John Benson, Betty Redfern, and Jeremy Roxbee Cox, 1--23. Oxford
University Press, 2006.

\item Simion, F., L. Regolin, and H. Bulf. ``A Predisposition for Biological
Motion in the Newborn Baby.'' \emph{Proceedings of the National Academy
of Sciences} 105, no. 2 (2008): 809--13.

\item Smith, David Woodruff. ``Phenomenology.'' Stanford Encyclopedia of
Philosophy. Stanford University, December 16, 2013.

\item Steele, Timothy. \emph{All the Fun's in How You Say a Thing an
Explanation of Meter and Versification}. Ohio Univ. Press, 1999.

\item Wimsatt, W., \& Beardsley, M. (1946). The intentional fallacy. The
Sewanee Review, 54(3), 468--488.
\end{itemize}
}
